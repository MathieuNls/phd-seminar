%!TEX root = research_proposal.tex

\chapter{Remaining Work to Complete the Thesis\label{chap:plan}}

In this chapter, we summarize the state of the research and the work that needs to be completed in order to finalize the thesis.
We proceed according to the contributions listed on Section\ref{sec:objective-thesis}.


\section{An analysis of Existing Techniques Aiming to Support Software Maintenance}

We need to create a classification for the tools and technics we observed to clearly show how they differ and why they were not adopted by industry.

\section{Pragamatic Software Maintenance}

We measured the efficiency for 60\% of our proposed framework for pragmatic software maintenance (three components out of five).
Our work for this contribution has already been featured in the following publications:

\begin{itemize}
	\item Nayrolles, M. , Hamou-Lhadj, W., Tahar, S. & Larsson, A. (2016). A Bug Reproduction Approach Based on Directed Model Checking and Crash Traces. Journal of Software: Evolution and Process. Wiley. 2016. (Accepted).
	\item Nayrolles, M. \& Hamou-Lhadj, W. BUMPER: A Tool to Cope with Natural Language Search of Millions Bugs and Fixes. In Proceeding of the International Conference on Software Analysis, Evolution, and Reengineering (SANER'16) - Tool Track, pages 649-652, 2016.
	\item Nayrolles, M. \& Hamou-Lhadj, W. BUMPER: Bug Metarepository Search Engine for Developers and Researchers. Consortium for Software Engineering Research Fall, 2015.
	\item Nayrolles, M. , Hamou-Lhadj, W., Tahar, S. & Larsson, A. JCHARMING : A Bug Reproduction Approach Using Crash Traces and Directed Model Checking. In Proceeding of the International Conference on Software Analysis, Evolution, and Reengineering (SANER'15), pages 101-110, 2015. (Best Paper Award).
\end{itemize}

We need to conduct additional experiments to measure the effiency of the two remaining components: {\tt RESEMBLE} and {\tt BIANCA}.

\section{A Taxonomy to Classify the Research on Software Maintenance}

We need to refine our statistical analysis for our taxonomy.
More specifically, we need to compare each bug type one by one in addition to the comparison we have already done.

\section{Publication Plan\label{sec:publication-plan}}

This section presents our planned publications over the course of the next years.

\begin{itemize}
	\item {\bf Publication 6}. {\tt PRECINCT} will be submitted to {\tt International Working Conference on Source Code Analysis and Manipulation, SCAM 2016}.
	\item {\bf Publication 7}. {\tt BIANCA} will be submitted to {\tt Journal of Software: Evolution and Process. 2016}.
	\item {\bf Publication 8}. {\tt RESEMBLE} will be submitted to {\tt International Conference Software Maintenance and Evolution, ICSME 2017}.
	\item {\bf Publication 9}. Our proposed bug taxonomy will be submitted to {\tt Empirical Software Engineering, ESE 2017}.
	\item {\bf Publication 10}. Our framework as a whole, {\tt BUMPER}, {\tt JCHARMING}, {\tt RESEMBLE} and {\tt BIANCA} will be submitted to {\tt Transaction of Software Engineering, TSE 2017}.
	\item {\bf Thesis}. In parallel to publications 9 and 10, I plan to write my Ph.D thesis.
\end{itemize}

Figure \ref{fig:planning} presents and overview of the publications planning and the relationship between the publications.

 \begin{figure}[h!]

 \centering
 \begin{ganttchart}
	  [
 inline
]{1}{30}
 \gantttitle{2014}{6}
 \gantttitle{2015}{6}
 \gantttitle{2016}{6}
 \gantttitle{2017}{6}
 \gantttitle{2018}{6}  \\
 \gantttitle{W}{2}
 \gantttitle{S}{2}
 \gantttitle{F}{2}
 \gantttitle{W}{2}
 \gantttitle{S}{2}
 \gantttitle{F}{2}
 \gantttitle{W}{2}
 \gantttitle{S}{2}
 \gantttitle{F}{2}
 \gantttitle{W}{2}
 \gantttitle{S}{2}
 \gantttitle{F}{2}
 \gantttitle{W}{2}
 \gantttitle{S}{2}
 \gantttitle{F}{2} \\


\ganttbar{$Courses$}{1}{4}

\\
\ganttbar{$JChar$}{5}{8}
\ganttbar{$JChar_2$}{12}{14}
\ganttbar{$Taxonomy$}{19}{22}
\\
\ganttbar{$Bumper$}{4}{11}
\ganttbar{$Pasmat$}{22}{25}
\\
\ganttbar{$Precinct$}{13}{16}
\\
\ganttbar{$Resemb$}{18}{20}
\ganttbar{$Thesis$}{25}{29}
\\
\ganttbar{$Bianca$}{15}{20}
\\


\ganttlink{elem0}{elem1}
\ganttlink{elem1}{elem2}
\ganttlink{elem2}{elem3}
\ganttlink{elem4}{elem2}
\ganttlink{elem4}{elem3}
\ganttlink{elem4}{elem6}
\ganttlink{elem6}{elem7}
\ganttlink{elem6}{elem9}

\ganttlink{elem6}{elem5}
\ganttlink{elem7}{elem5}
\ganttlink{elem9}{elem5}
\ganttlink{elem3}{elem5}
\ganttlink{elem5}{elem8}

 \end{ganttchart}


 \caption{Provisional Publication Planning\label{fig:planning}}
 \end{figure}
