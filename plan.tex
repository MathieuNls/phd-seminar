%!TEX root = research_proposal.tex

\chapter{Research Plan\label{chap:plan}}

In this chapter, we present the current state of research in section \ref{sec:current-state}, our current contributions in section \ref{sec:current-state}. Section \ref{sec:publication-plan} present our publication plan.

\section{Current State of Research\label{sec:current-state}}

Chapter \ref{chap:methodology} represents the current state of our research. The parts that are already complete are :

\begin{itemize}
	\item To create a taxonomy of bug (section \ref{sec:taxo}).
	\item To make software artifacts such as source code, issues, tasks, comments and more searchable using natural language and not structured query language (section \ref{sec:BUMPER}).
	\item To efficiently reproduce field crashes in lab environment without raising any privacy concerns (section \ref{sec:JCHARMING}).
	\item To sequentially mine co-changes (section \ref{sec:RESEMBLE}).
	\item To normalize source code (sections \ref{sec:RESEMBLE} and \ref{sec:BIANCA}).
	\item To efficiently compare normalized source code (sections \ref{sec:RESEMBLE} and \ref{sec:BIANCA}).
	\item To raise warning when a normalized source code matches another normalized source code known to have introduce a defect in an application (section \ref{sec:BIANCA}).
\end{itemize}

\section{Current Contribution\label{sec:current-state}}

We already have one contribution referred in an international conference \cite{Nayrolles2015}:
\\ 

\noindent Mathieu Nayrolles, Abdelwahab Hamou-Lhadj, Tahar Sofiene, and Alf Larsson. JCHARMING : A Bug Reproduction Approach Using Crash Traces and Directed Model Checking. In SANER’15, pages 101-110, 2015. (Best Paper Award).
\\ 

This paper was published in (International Conference on Software Analysis, Evolution, and Reengineering). SANER is a merge of WCRE (Working Conference on Reverse Engineering) and CSMR (European Conference on Software Maintenance and Reengineering). 

\todo{Is bumper-app.com a contribution w/out its paper ?}

\section{Plan for short term work\label{sec:planning}}

As short term work, we want to complete the following parts:

\begin{itemize}
	\item Recommend code in order to improve a sub-optimum code.
	\item Recommend code in order to complete an incomplete piece of code.
	\item Develop IDE plugins
	\item Develop new parsers (Github, Sourceforge) in order to improve:
	\begin{itemize}
		\item The coverage of {\tt BUMPER}.
		\item The precision of {\tt RESEMBLE} and {\tt BIANCA}.
		\item The accuracy of our Bug Taxonomy.
	\end{itemize}
\end{itemize}

\section{Publication Plan\label{sec:publication-plan}}

This section presents our planned publications over the course of the next years. 
Figure \ref{fig:planning} shows how publications relate to each other and an attempt planning. 
Depending on time constraints and possible invitations to journal extension, publication can be removed or added during the next years.

\begin{itemize}
	\item {\bf Publication 2}. {\tt BUMPER} presented in section \ref{sec:BUMPER} will be submitted  as a tool paper to {\tt Automated Software Engineering, ASE 2015}.
	\item {\bf Publication 3}. Being awarded the Best Paper Award at SANER'15 \cite{Nayrolles2015}, we were invited to extend our work to a journal version. Accordingly, we intend to perform more extensive experiments and to describe in depth our directed model checking engine and our junit test case generation algorithm. ({\tt Journal of Software: Evolution and Process 2016}).
	\item {\bf Publication 4}. {\tt BIANCA} will be submitted to {\tt International Conference on Software Engineering, ICSE 2016}.
	\item {\bf Publication 5}. A joint work with McGill university that predict co-changes by mining execution traces produced by instrumented software will be submitted to {\tt International Conference on Software Engineering, ICSE 2016}.
	\item {\bf Publication 6}. A derivate of {\tt BIANCA} that detect clones at commit time will be submitted to {\tt International Conference on Software Analysis, Evolution, and Reengineering, SANER 2016}
	\item {\bf Publication 7}. {\tt RESEMBLE} will be submitted to {\tt International Conference Software Maintenance and Evolution, ICSME 2017}
	\item {\bf Publication 8}. Our proposed bug taxonomy will be submitted to {\tt Transaction of Software Engineering, TSE 2017}
	\item {\bf Publication 9}. Our ecosystem as a whole, {\tt BUMPER}, {\tt JCHARMING}, {\tt RESEMBLE} and {\tt BIANCA} will be submitted to {\tt Transaction of Software Engineering, TSE 2017}.
	\item {\bf Thesis}. In parallel to publication 8 and 9, I plan to write my Ph.D thesis.
\end{itemize}

\begin{figure}[h!]

\centering 

\begin{ganttchart}[
hgrid,
vgrid,
inline]{1}{27}
\gantttitle{2014}{9} 
\gantttitle{2015}{9} 
\gantttitle{2016}{9} \\
\gantttitle{T1}{3}
\gantttitle{T2}{3}
\gantttitle{T3}{3}
\gantttitle{T1}{3}
\gantttitle{T2}{3}
\gantttitle{T3}{3}
\gantttitle{T1}{3}
\gantttitle{T2}{3}
\gantttitle{T3}{3} \\


\ganttbar{$Taxonomy$}{7}{11}
\ganttbar{$Taxonomy$}{19}{23} \\
\ganttbar{$Bumper$}{11}{13}
\ganttbar{$Resemb$}{17}{19} \\
\ganttbar{$Bianca$}{10}{14}
\ganttbar{$Bianca_2$}{16}{18}
\ganttbar{$Thesis$}{22}{27}\\
\ganttbar{$Courses$}{1}{6}
\ganttbar{$McGill$}{11}{14}
\ganttbar{$Eco$}{20}{22} \\
\ganttbar{$JChar$}{5}{7}
\ganttbar{$JChar_2$}{12}{14}


\ganttlink{elem0}{elem2}
\ganttlink{elem0}{elem1}
\ganttlink{elem2}{elem3}
\ganttlink{elem2}{elem4}
\ganttlink{elem4}{elem5}
\ganttlink{elem5}{elem9}
\ganttlink{elem9}{elem6}
\ganttlink{elem1}{elem6}
\ganttlink{elem3}{elem9}
\ganttlink{elem4}{elem8}
\ganttlink{elem10}{elem11}
\ganttlink{elem11}{elem9}
\ganttlink{elem7}{elem10}
\ganttlink{elem10}{elem0}

\end{ganttchart}


\caption{Provisional Publication Planning\label{fig:planning}}
\end{figure}
\todo{This is a very high level planning. Is it enough ?}

