%!TEX root = ../research_proposal.tex


{\tt BIANCA} (Bug Insertion ANticipation by Clone Analysis at merge time) is an approach that we propose and which aims to prevent the insertion of bugs at commit-time. Many tools exist to prevent a developer to ship {\it bad} code \cite{Dangel2000,Hovemeyer2007,Moha2010} or to identify {\it bad} code after executions (e.g in test or production environment) \cite{Nayrolles,Nayrolles2013a}. However, these tools rely on metrics and rules to statically and/or dynamically identify sub-optimum code. {\tt BIANCA} is different than the approaches presented in the previous sections because it mines and analyses the change patterns in commits and matches them against past commits known to have introduced a defect in the code (or that have just been replaced by better implementation).
Also, {\tt BIANCA} is an offline approach that is triggered by a merge request.
When  maintainers see that their work are ready to be integrated with the main branch, they open a merge request\footnote{Also known as pull request}.
Merging a task branch is not an instantaneous process as the code need to pass code review.
{\tt BIANCA} leverages this {\it down} time to perform a complete history check on all projects contained in {\tt BUMPER}.

Figure \ref{fig:bianca-approach} presents an overview of our approach.

\begin{figure}[h!]
  \centering
    \includegraphics{media/bianca-approach.png}
    \caption{The BIANCA Approach
    \label{fig:bianca-approach}}
\end{figure}

{\tt BIANCA} builds a model where each issue is represented by three versions of the same file.
These three versions are stored in {\tt BUMPER}.
The first version $n$ is called the {\it stable state} because the code of this version was used to fix an issue.
The $n-1$ version, however, is called the {\it unstable state} as it was marked as containing an issue.
Finally, the third version is called the {\it before state} and represents the file before the introduction of the bug.
Hereafter, we refer to the {\it before state} as $n-2$.
{\tt BIANCA} extracts the change patterns form $n-2$ to $n-1$ and from $n-1$ to $n$.
It aslo generates the changes to go from $n-2$ to $n$.

When a developer commits new modifications, {\tt BIANCA} extracts the change pattern from the version $n_{dev}$ (current version) and $n-1_{dev}$ (version before modification) of the developer's source code and compares this change patterns to known $n-2$ to $n-1$ patterns.
If $n_{dev}$ to $n-1_{dev}$ matches a $n-2$ to $n-1$ then it means that the developer is inserting a known defect in the source code.
In such a case, {\tt BIANCA} will propose the related $n-1$ to $n$ patterns to the developer, so s/he could improve the source code and will show the related $n-2$ for the $n$ pattern so the developer will learn how to s/he should have modified the code in the first place.

Moreover, if the issue was previously reproduced by {\tt JCHARMING}, then {\tt BIANCA} will display the steps to reproduce it.


To extract the change patterns and compare them, we use the same technique as the one presented in section \ref{sub:Extract and Save Blocks}.
The third and the fourth normalizations are removing all {\it less} important calls in the normalization one and two of {\tt RESEMBLE}.
We classify a call as less-important if, for example, it only does display-related functionalities such as generating HTML or printing something to the console. Finally, the fourth normalization will transform the code to an intermediate language of our own that will allow us to compare source code implemented in different programming languages.

Then, as in {\tt RESEMBLE}, if the LCS is above a user-defined threshold, then a warning is raised by {\tt BIANCA} alerting the developer that the commit is suspected to insert a defect.
The given defect is shown to the developer and can either force the commit if s/he don't find the warning relevant or abort the commit.

We believe that the warning, alongside the previously mined change patterns and steps to reproduce the suspected default --- provided by {\tt JCHARMING}, if available --- will statisfy developers in terms of actionable inteligence.
Thus, {\tt BIANCA} could succeed, where other tools failed, at being used in industrial environment	\cite{Lewis2013}.


\section{Early experiments}

We have assessed the efficiency of {\tt BIANCA} with the same datasets we used to build our bug taxonomy proposed in Section \ref{chap:taxonomy}.

\begin{table}[h]
\begin{center}
\begin{tabular}{@{}c|c|c|c|c@{}}
\textbf{Dataset} & \textbf{Fixed Issues} & \textbf{Commit} & \textbf{Files} & \textbf{Projects} \\ \hline \hline
Netbeans         & 53,258          & 122,632     & 30,595         & 39                \\
Apache           & 49,449          & 106,366     & 38,111         & 349               \\
Total            & 102,707         & 229,153     & 68,809         & 388               \\ \hline \hline

\end{tabular}
\end{center}

\caption{Datasets\label{table:datasets-bianca}}
\end{table}


We ran two different experiments using the two first normalizations we described in Section \ref{sec:bianca-approach}. Both experiments consider only a few months of history, from April to August 2008. While this could hinder the accuracy of our results, these five-month  history data contain 167,597 commits related to bug fixes. We believe  that the results to be representative.

The first experiment yields the result presented in Figure \ref{fig:bianca-exp-1}.

\begin{figure}[h!]
  \centering
    \includegraphics[scale=0.55]{media/bianca-13.png}
    \caption{{\tt BIANCA} warnings from April to August 2008 using the first normalization.
    \label{fig:bianca-exp-1}}
\end{figure}

With the first normalization, {\tt BIANCA} raised 69,519 warnings out of 167,597 (41.5\%) analysed commits.
Out of these 69,519, 13.4\% turned out to be false positives. A false positive is a healthy commit that has been tagged as introducing a bug by {\tt BIANCA}.
However, false positives have to be dealt with carefully in this study as the commit might have introduced a bug but the bug has not been reported yet.

In our second experiment, we used the second normalization and {\tt BIANCA} raised 83,627 warnings out of 167,597 (48.89\%) commit we analyze. However, the false positive rate increased to 21\%. Figure \ref{fig:bianca-exp-2} shows the results.

\begin{figure}[h!]
  \centering
    \includegraphics[scale=0.55]{media/bianca-20.png}
    \caption{{\tt BIANCA} warnings from April to August 2008 using the second normalization.
    \label{fig:bianca-exp-2}}
\end{figure}

{\tt BIANCA} experiments are still in their early stages and we are still trying to improve our normalization in order to reduce the false positive rate.
