%!TEX root = ../research_proposal.tex

% \section{PRECINCT: PREventing Clones INsertion at Commit Time}

In this section, we present PRECINCT (PREventing Clones INsertion at Commit Time) that focuses on preventing the insertion of clones at commit time, i.e., before they reach the central code repository. PRECINCT is an online clone detection technique that relies on the use of pre-commit hooks capabilities of modern source code version control systems.
PRECINCT intercepts this modification and analyses its  content to see whether a suspicious clone has been introduced or not.
A flag is raised if a code fragment is suspected to be a clone of an existing code segment.
In fact, PRECINCT, itself, can be seen as a pre-commit hook that detects clones that might have been inserted in the latest changes with regard to the rest of the source code.
This said, only a fraction of the code is analysed, making PRECINCT efficient compared to leading  clone detection techniques such as NICAD (Accurate Detection of Near-miss Intentional Clones) \cite{Cordy2011}.
Moreover, the detected clones are presented using a classical `diff' output that developers are familiar with.
PRECINCT is also well integrated with the workflow of the developers since it is used in conjunction with a source code version control systems such as Git\footnote{https://git-scm.com/}.

In this study, we focus on Type 3 clones as they are more challenging to detect. Since Type 3 clones include Type 1 and 2 clones, then these types could be detected separately by PRECINCT as well.

PRECINCT aims to prevent clone insertion while integrating
the clone detection process in a transparent manner
in the day-to-day maintenance process.
This way, software developers do not have to resort to external tools to remove clones after they are inserted such as the one presented in Section \ref{sec:rel-clones}.
Our approach operates at commit time, notifying software developers of possible clones as they commit their code.

We evaluated the effectiveness of PRECINCT using precision and recall on three systems, developed independently and written in both C and Java. The results show that PRECINCT prevents Type 3 clones to reach the final source code repository with an average accuracy of 97.7\%.

\section{The PRECINCT Approach}
\label{sec:The PRECINCT Approach}

\begin{figure*}
  \centering
    \includegraphics[width=\textwidth]{media/approach.png}
    \caption{ Overview of the PRECINCT Approach.\label{fig:precinct-approach}}
\end{figure*}

The PRECINCT approach is composed of six steps.
The first and last steps are typical steps that a developer would do when committing code.
Indeed, the first step is the commit step where developers send their latest changes to the central repository and the last step is the reception of the commit by the central repository.
The second step is the pre-commit hook, which kicks in as the first operation when one wants to commit.
The pre-commit hook has access to the changes in terms of files that have been modified, more specifically, the lines that have been modified. The modified lines of the files are sent to TXL\cite{Cordy2006a} for block extraction. Then, the blocks are compared to previously extracted blocks in order to identify candidate clones using the comparison engine of NICAD\cite{Cordy2011}. We chose NICAD engine because it has been shown to provide high accuracy \cite{Cordy2011}. The tool is also readily available, easy to use, customizable, and works with TXL. Note, however, that PRECINCT can also work with other engines for comparing code fragments.
Finally, the output of NICAD is further refined and presented to the user for decision.
These steps are discussed in more detail in the following subsections.

\section{PRECINCT Pre-Commit Hook}
\label{sub:Pre-Commit Hook}

Depending on the exit status of the hook, the commit will be aborted and not pushed to the central repository.
Also, developers can choose to ignore the pre-hook. In Git, for example, they will need to use the command \texttt{git commit --no-verify} instead of \texttt{git commit}.
This can be useful in case of an urgent need for fixing a bug where the code has to reach the central repository as quickly as possible.
Developers can do things like check for code style, check for trailing white spaces (the default hook does exactly this), or check for appropriate documentation on new methods.

PRECINCT is a set of bash scripts where the entry point of these scripts lies in the pre-commit hooks. Pre-commit hooks are easy to create and implement as depicted in Listing~\ref{gitprehook}.
This pre-hook is shipped with Git, a popular version control system.
Note that even though we use Git as the main version control to present PRECINCT, we believe that the techniques presented in this section are readily applicable to other version control systems.
In Listing~\ref{gitprehook}, from lines 3 to 11, the script identifies if the commit is the first one in order to select the revision to work against.
Then, in Lines 18 and 19, the script checks for trailing whitespace and fails if any are found.

For PRECINCT to work, we just have to add the call to our script suite instead or in addition of the whitespace check.

\noindent\begin{minipage}{0.90\linewidth}

  \lstinputlisting[language=Bash, firstnumber=1, numbers=right, stepnumber=1,leftmargin=30, label=gitprehook, caption=Git Pre-Commit Hook Sample]{media/pre-commit.sample}

\end{minipage}


\section{Extract and Save Blocks}
\label{sub:Extract and Save Blocks}

A block is a set of consecutive lines of code that will be compared to all other blocks in order to identify clones.
To achieve this critical part of PRECINCT, we rely on TXL\cite{Cordy2006a}, which is a first-order functional programming over linear term rewriting, developed by Cordy et al.\cite{Cordy2006a}.
For TXL to work, one has to write a grammar describing the syntax of the source  language and the transformations needed. TXL has three main phases: \textit{parse, transform}, \textit{unparse}.
In the parse phase, the grammar controls not only the input but also the output form.
Listing~\ref{txlsample} --- extracted from the official documentation\footnote{http://txl.ca} --- shows a grammar matching an \textit{if-then-else} statement in C with some special keywords: [IN] (indent), [EX] (exdent) and [NL] (newline) that will be used for the output form.

\noindent\begin{minipage}{0.90\linewidth}

  \lstinputlisting[language=Bash, firstnumber=1, numbers=right, stepnumber=1,leftmargin=30, label=txlsample, caption=Txl Sample Sample]{media/txl.sample}

\end{minipage}

Then, the \textit{transform} phase will, as the name suggests, apply transformation rules that can, for example, normalize or abstract the source code. Finally, the third phase of TXL,  called \textit{unparse}, unparses the transformed parsed input in order to output it.
Also, TXL supports what the creators call Agile Parsing\cite{Dean}, which allow developers to redefine the rules of the grammar and, therefore, apply different rules than the original ones.


PRECINCT takes advantage of that by redefining the blocks that should be extracted for the purpose of clone comparison, leaving out the  blocks that are out of scope.
More precisely, before each commit, we only extract the blocks belonging to the modified parts of the source code.
Hence, we only process, in an incremental manner, the latest modification of the source code instead of the source code as a whole.

We have selected TXL for several reasons. First, TXL is easy to install and to integrate with the normal workflow of a developer.
Second, it was relatively easy to create a grammar that accepts commits as input.
This is because TXL is shipped with C, Java, Csharp, Python and WSDL grammars that define all the particularities of these languages, with the ability to customize these grammars to accept changesets (chunks of the modified source code that include the added, modified, and deleted lines) instead of the whole code.

\begin{algorithm}
 \KwData{$Changeset[]$ changesets\;
 $Block[]$ prior\_blocks\;
 $Boolean$ compare\_history\;
 }
 \KwResult{Up to date blocks of the systems}
 \For{$i \leftarrow 0$ \KwTo$size\_of~changesets$}{
    Block[] blocks $\leftarrow$ $extract\_blocks(changesets)$\;
    \For{$j \leftarrow 0$ \KwTo$size\_of~blocks$}{
       \If{not $compare\_history$ AND $blocks[j]$ overrides one of $prior\_blocks$}{
          delete $prior\_block$\;
       }
       write $blocks[j]$\;
    }
 }

 \SetKwProg{myproc}{Function}{ $~extract\_blocks(Changeset~cs)$}{}
   \myproc{\proc{}}{

   \uIf{$cs~is~unbalanced~right$}{$cs \leftarrow expand\_left(cs)$\;}

   \ElseIf{$cs~is~unbalanced~left$}{$cs \leftarrow expand\_right(cs)$\;}

   \nl\KwRet$txl\_extract\_blocks(cs)$\;
   }


 \caption{Overview of the Extract Blocks Operation\label{alg:extract}}
\end{algorithm}

\noindent\begin{minipage}{0.90\linewidth}

  \lstinputlisting[language=Bash, firstnumber=1, numbers=right, stepnumber=1,leftmargin=30, label=commitsample, caption=Changeset c4016c of monit]{media/commit.sample}

\end{minipage}


Algorithm~\ref{alg:extract} presents an overview of the ``extract" and ``save" blocks operations of PRECINCT. This algorithm receives as arguments, the changesets, the blocks that have been previously extracted and a boolean named compare\_history.
Then, from Lines 1 to 9 lie the $for$ loop that iterates over the changesets. For each changeset (Line 2), we extract the blocks by calling the $~extract\_blocks(Changeset~cs)$ function.
In this function, we expand our changeset to the left and to the right in order to have a complete block.

As depicted by Listing~\ref{commitsample}, changesets contain only the modified chunk of code and not necessarily complete blocks. Indeed, we have a block from Line 3 to Line 6 and deleted lines from Line 8 to 14.
However, in Line 7 we can see the end of a block, but we do not have its beginning. Therefore, we need to expand the changeset to the left in order to have syntactically correct blocks.
We do so by checking the block's beginning and ending (using \{ and \}) in C for example.
Then, we send these expanded changesets to TXL for block extraction and formalization.

For each extracted block, we check if the current block overrides (replaces) a previous block (Line 4).
In such a case, we delete the previous block as it does not represent the current version of the program anymore (Line 5).
Also, we have an optional step in PRECINCT defined in Line 4. The compare\_history is a condition to delete overridden blocks.

We believe that deleted blocks have been deleted for a good reason (bug, default, removed features, \ldots) and if a newly inserted block matches an old one, it could be worth knowing in order to improve the quality of the system at hand.
This feature is deactivated by default.

In summary, this step receives the files and lines, modified by the latest changes made by the developer and produces an up to date block representation of the system at hand in an incremental way.
The blocks are analysed in the next step to discover potential clones.

\section{Compare Extracted Blocks}
\label{sub:Compare Extracted Blocks}

In order to compare the extracted blocks and detect potential clones, we can only resort to text-based techniques.
This is because lexical and syntactic analysis approaches (alternatives to text-based comparisons) would require a complete program to work, a program that compiles.
In the relatively wide-range of tools and techniques that exist to detect clones by considering code as text\cite{Johnson1993,Johnson1994,Marcus,Manber1994,StephaneDucasse,Wettel2005}, we selected NICAD as the main text-based method for comparing clones \cite{Cordy2011} for several reasons.
First, NICAD is built on top of TXL, which we also used in the previous step.
Second, NICAD is able to detect all Types 1, 2 and 3 software clones.

NICAD  works in three phases: \textit{Extraction}, \textit{Comparison} and \textit{Reporting}. During the \textit{Extraction} phase all potential clones are identified, pretty-printed, and extracted.
We do not use the \textit{Extraction} phase of NICAD as it has been built to work on programs that are syntactically correct, which is not the case for changesets.
We replaced NICAD's \textit{Extraction} phase with our own scripts, described in the previous section.

In the \textit{Comparison} phase, extracted blocks are transformed, clustered and compared in order to find potential clones.
Using TXL sub-programs, blocks go through a process called pretty-printing where they are stripped of formatting and comments.
When code fragments are cloned, some comments, indentation or spacing are changed according to the new context where the new code is used. This pretty-printing process ensures that all code will have the same spacing and formatting, which renders the comparison of code fragments easier.
Furthermore, in the pretty-printing process, statements can be broken down into several lines.
Table~\ref{tab:pretty-printing} shows how this can improve the accuracy of clone detection with three \texttt{for} statements, \texttt{ for (i=0; i<10; i++)}, \texttt{for (i=1; i<10; i++)} and \texttt{ for (j=2; j<100; j++)}.
The pretty-printing allows NICAD to detect Segments 1 and 2 as a clone pair because only the initialization of $i$ changed.
This specific example would not have been marked as a clone by other  tools we tested such as Duploc\cite{Ducasse1999}.
In addition to the pretty-printing, code can be normalized and filtered to detect different classes of clones and match user preferences.

\begin{table}[]
\centering
\caption{Pretty-Printing Example\cite{Iss2009}}
\label{tab:pretty-printing}
\resizebox{0.5\textwidth}{!}{%
\begin{tabular}{l|l|l|l|l|l}
Segment 1          & Segment 2          & Segment 3           & S1 \& S2 & S1 \& S3 & S2 \& S3 \\ \hline \hline
for (              & for (              & for (               & 1        & 1        & 1        \\
i = 0;             & i = 1;             & j = 2;              & 0        & 0        & 0        \\
i \textgreater 10; & i \textgreater 10; & j \textgreater 100; & 1        & 0        & 0        \\ 
i++)               & i++)               & j++)                & 1        & 0        & 0        \\ \hline \hline
\multicolumn{3}{c|}{Total Matches}                            & 3        & 1        & 1        \\ \hline
\multicolumn{3}{c|}{Total Mismatches}                         & 1        & 3        & 3
\end{tabular}
}
\end{table}


Finally, the extracted, pretty-printed, normalized and filtered blocks are marked as potential clones using a Longest Common Subsequence (LCS) algorithm\cite{Hunt1977}. Then, a percentage of unique statements can be computed and, depending on a given threshold (see Section~\ref{sec:Experimentations}), the blocks are marked as clones.

The last step of NICAD, which acts as our clone comparison engine, is the \textit{reporting}. However, to prevent PRECINCT from outputting a large amount of data (an issue that many clone detection techniques face), we  implemented our own reporting system, which is also well embedded with the workflow of developers. This reporting system is the subject of the next section.

As a summary, this step receives potentially expanded and balanced blocks from the extraction step.
Then, the blocks are pretty-printed, normalized, filtered and fed to an LCS algorithm in order to detect potential clones.
Moreover, the clone detection in PRECINCT is less intensive than NICAD because we only compare the latest changes with the rest of the program instead of comparing all the blocks with each other.

\section{Output and Decision}
\label{sub:Output and Decision}

In this final step, we report the result of the clone detection at commit time with respect to the latest changes made by the developer. The process is straightforward. Every change made by the developer goes through the previous steps and is checked for the introduction of potential clones. For each file that is suspected to contain a clone, one line is printed to the command line with the following options: (I) Inspect, (D) Disregard, (R) Remove from the commit as shown by Figure \ref{fig:hook}. In comparison to this simple and interactive output, NICAD outputs each and every detail of the detection result such as the total number of potential clones, the total number of lines, the total number of unique line text chars, the total number of unique lines, and so on. We think that so many details might make it hard for developers to react to these results. A problem that was also raised by Johnson et al. \cite{Johnson2013} when examining bug detection tools.
Then the potential clones are stored in XML files that can be viewed using an Internet browser or a text editor.

\begin{figure}
  \centering
    \includegraphics[width=0.48\textwidth]{media/commit.png}
    \caption{PRECINCT output when replaying commit \texttt{710b6b4} of the Monit system used in the case study.\label{fig:hook}}
\end{figure}

(I) Inspect will cause a diff-like visualization of the suspected clones while (D) disregard will simply ignore the finding.
To integrate PRECINCT in the workflow of the developer we also propose the  remove option (R). This option will simply remove the suspected file from the commit that is about to be sent to the central repository.
Also, if the user types an option key twice, e.g., II, DD or RR, then the option will be applied to all files.
For instance, if the developer types DD at any point, the PRECINCT's results will be disregarded and the commit will be allowed to go through. We believe that this simple mechanism will encourage developers to use PRECINCT like they would use any other feature of Git (or any other version control system).


\section{Case Study}
\label{sec:Experimentations}

In this section, we show the effectiveness of PRECINCT for
detecting clones at commit time in three open source systems\footnote{The programs used and instructions to reproduce the experiments are made available for download from https://research.mathieu-nayrolles.com/precinct/}.

The aim of the case study is to answer the following question: \textit{Can we detect clones at commit time, i.e., before they are inserted in the final code, if so, what would be the accuracy compared to a traditional clone detection tool such as NICAD?}

\subsubsection{Target Systems}
\label{sub:Target Systems}

Table~\ref{tab:sut} shows the systems used in this study and their characteristics in terms of the number files they contain and the size in KLoC (Kilo Lines of Code). We also include the number of revisions used for each system and the programming language in which the system is written.

\begin{table}[]
\centering
\caption{List of Target Systems in Terms of Files and Kilo Line of Code (KLOC) at current version and Language}
\label{tab:sut}
\resizebox{0.5\textwidth}{!}{%
\begin{tabular}{c|c|c|c|c}
SUT        & Revisions & Files & KLoC & Language \\ \hline\hline
Monit      & 826       & 264   & 107  & C        \\ \hline
Jhotdraw   & 735       & 1984  & 44   & Java     \\ \hline
dnsjava    & 1637      & 233   & 47   & Java     \\ \hline
\end{tabular}
}
\end{table}

Monit\footnote{https://mmonit.com/monit/} is a small open source utility for managing and monitoring Unix systems.
Monit is used to conduct automatic maintenance and repair and supports the ability to identify causal actions to detect errors.
This system is written in C and composed of 826 revisions, 264 files, and the latest version has 107 KLoC.
We have chosen Monit as a target system because it was one of the systems NICAD was tested on.

JHotDraw\footnote{http://www.jhotdraw.org/} is a Java GUI framework for technical and structured graphics.
It has been developed as a ``design exercise''. Its design relies heavily on the use of design patterns. JHotDraw is composed of 735 revisions, 1984 files, and the latest revision has 44 KLoC. It is written in Java and it is often used by researchers as a test bench. JHotDraw was also used by NICAD's developers to evaluate their approach.

Dnsjava\footnote{http://www.dnsjava.org/} is a tool for implementing the DNS (Domain Name Service) mechanisms in Java.
This tool can be used for queries, zone transfers, and dynamic updates.
It is not as large as the other two, but it still makes an interesting case subject because it has been well maintained for the past decade. Also, this tool is used in many other popular tools such as Aspirin, Muffin and
Scarab. Dnsjava is composed of 1637 revisions, 233 files, the latest revision contains 47 KLoC.

\section{Process}
\label{sub:Process}


Figure \ref{fig:precinct-branching} shows the process we followed to validate the effectiveness of PRECINCT.

\begin{figure}
  \centering
    \includegraphics[width=0.3\textwidth]{media/branch.png}
    \caption{PRECINCT Branching.\label{fig:precinct-branching}}
\end{figure}

As our approach relies on commit pre-hooks to detect possible clones during the development process (more particularly at commit time), we had to find a way to \textit{replay} past commits. To do so, we  \textit{cloned} our test subjects, and then created a new branch called \textit{PRECINCT\_EXT}.
When created, this branch is reinitialized at the initial state of the project (the first commit) and each commit can be replayed as they have originally been. At each commit, we store the time taken for PRECINCT to run as well as the number of detected clone pairs. We also compute the size of the output in terms of the number of lines of text output by our method. The aim is to reduce the output size to help software developers interpret the results.

To validate the results obtained by PRECINCT, we needed to use a reliable clone detection approach to extract clones from the target systems and use these clones as a baseline for comparison. For this, we turned to NICAD because of its popularity, high accuracy, and availability \cite{Cordy2011}, as discussed before. This means, we run NICAD on the revisions of the system to obtain the clones then we used NICAD clones as a baseline for comparing the results obtained by PRECINCT.

It may appear strange that we are using NICAD to validate our approach, knowing that our approach uses NICAD's code comparison engine. In fact, what we are assessing here is the ability for PRECINCT to detect clones at commit time using changsets. The major part of PRECINCT is the ability to  intercept code changes and build working code blocks that are fed to a code fragment engine (in our case NICAD's engine). PRECINCT can be built on the top of any other code comparison engine.

We show the result of detecting Type 3 clones with a maximum line difference of 30\% as discussed in Table~\ref{tab:result}. As discussed in the introductory section, we chose to report on Type 3 clones because they are more challenging to detect than Type 1 and 2. PRECINCT detects Type 1 and 2 too so does NICAD. For the time being, PRECINCT is not designed to detect Type 4 clones. These clones use different implementations. Detecting Type 4 clones is part of future work.

We assess the performance of PRECINCT in terms of precision (Equation 1) and recall (Equation 2).
Both precision and recall are computed by considering  NICAD's results as a baseline.
We also compute F$_{1}$-measure (Equation 3), i.e., the weighted average of precision and recall, to measure the accuracy of PRECINCT.

\begin{equation}
precision = \frac{|\{ NICAD_{detection} \} \cap \{ PRECINCT_{detection} \} |}{| \{ PRECINCT_{detection} \}|}
\end{equation}

\begin{equation}
recall = \frac{|\{ NICAD_{detection} \} \cap \{ PRECINCT_{detection} \} |}{| \{ NICAD_{detection} \}|}
\end{equation}

\begin{equation}
F_1-measure = 2 * \frac{precision * recall}{precision + recall}
\end{equation}


\begin {figure*}%[!hbtp]
\centering
\begin{tikzpicture}
    \begin{axis}[
      scale only axis, % The height and width argument only apply to the actual axis
    height=5cm,
    width=0.9\textwidth,
    grid = both,
    minor tick num=1,
    xlabel={Revisions},
    ylabel={Clones},
    xmin=0]
    \addplot table [smooth,red,mark=none,x=revision,y=clones,col sep=comma] {methodology/data/monit};
    \addlegendentry{NICAD Detection}

    \addplot table [smooth,blue,mark=none,x=revision,y=detected,col sep=comma] {methodology/data/monit};
    \addlegendentry{PRECINCT Detection}

    \addplot table [smooth,red,mark=none,x=revision,y=prevented,col sep=comma] {methodology/data/monit};
    \addlegendentry{Remaining Clones}

    \end{axis}
    \end{tikzpicture}
    \caption{Monit clone detection over revisions\label{fig:r1}}
\end{figure*}

\begin {figure*}%[!hbtp]
\centering
\begin{tikzpicture}
    \begin{axis}[
      scale only axis, % The height and width argument only apply to the actual axis
    height=5cm,
    width=0.9\textwidth,
    grid = both,
    minor tick num=1,
    xlabel={Revisions},
    ylabel={Clones},
    xmin=0]
  \addplot table [smooth,red,mark=none,x=revision,y=clones,col sep=comma] {methodology/data/jhotdraw};
      \addlegendentry{NICAD Detection}

    \addplot table [smooth,green,mark=none,x=revision,y=detected,col sep=comma] {methodology/data/jhotdraw};
    \addlegendentry{PRECINCT Detection}

    \addplot table [smooth,red,mark=none,x=revision,y=prevented,col sep=comma] {methodology/data/jhotdraw};
    \addlegendentry{Remaining Clones}

    \end{axis}
    \end{tikzpicture}
    \caption{JHotDraw clone detection over revisions\label{fig:r2}}
\end{figure*}

\begin {figure*}%[!hbtp]
\centering
\begin{tikzpicture}
    \begin{axis}[
      scale only axis, % The height and width argument only apply to the actual axis
    height=5cm,
    width=0.9\textwidth,
    grid = both,
    minor tick num=1,
    xlabel={Revisions},
    ylabel={Clones},
    xmin=0]
  \addplot table [smooth,red,mark=none,x=revision,y=clones,col sep=comma] {methodology/data/dnsjava};
    \addlegendentry{NICAD Detection}

    \addplot table [smooth,green,mark=none,x=revision,y=detected,col sep=comma] {methodology/data/dnsjava};
    \addlegendentry{PRECINCT Detection}

    \addplot table [smooth,red,mark=none,x=revision,y=prevented,col sep=comma] {methodology/data/dnsjava};
    \addlegendentry{Remaining Clones}

    \end{axis}
    \end{tikzpicture}
    \caption{Dnsjava clone detection over revisions\label{fig:r3}}
\end{figure*}


\section{Results}
\label{sub:Results}

Figures~\ref{fig:r1},~\ref{fig:r2},~\ref{fig:r3} show the results of our study in terms of clone pairs that are detected per revision for our three subject systems: Monit, JHotDraw and Dnsjava. We used as baseline for comparison the clone pairs detected by NICAD. The blue line shows the clone detection performed by NICAD. The red line shows the clone pairs detected by PRECINCT. The brown line shows the clone pairs that have been missed by PRECINCT. As we can quickly see, the blue and red lines almost overlap, which indicates a good accuracy of the PRECINCT approach.

\begin{table*}[]
\centering
\caption{Overview of PRECINCT's results in terms of precision, recall, F$_{1}$-measure, execution time and output reduction.}
\label{tab:result}

\begin{tabular}{c|c|c|c|c|c||c|c|c}
         & NICAD
         & PREC.
         & Prec.
         & Recall
         & F1
         & \begin{tabular}[c]{@{}c@{}}NICAD'\\  Time\end{tabular}
         & \begin{tabular}[c]{@{}c@{}}PREC.'s\\  Time\end{tabular}
         & \begin{tabular}[c]{@{}c@{}}Output \\  Reduc.\end{tabular} \\ \hline\hline
Monit    & 128   & 123      & 96.1\%    & 100\%  & 98\%       & 2.2s                                                                     & 0.9s                                                                         & 88.3\%                                                              \\ \hline
JHotDraw & 6599  & 6490     & 98.3\%    & 100\%  & 99.1\%     & 5.1s                                                                     & 1.7s                                                                         & 70.1\%                                                              \\ \hline
DnsJava  & 273   & 226      & 82.8\%    & 100\%  & 90.6\%     & 1.8s                                                                     & 1.1s                                                                         & 88.6\%                                                              \\ \hline\hline
Total    & 7000  & 6839     & 97.7\%    & 100\%  & 98.8\%     & 3s                                                                       & 1.2s                                                                         & 83.4\%                                                              \\ \hline
\end{tabular}

\end{table*}


Table~\ref{tab:result} summarizes PRECINCT's results in terms of precision, recall, F$_{1}$-measure, execution time and output reduction.
The first version of Monit contains 85 clone pairs and this number stays stable until Revision 100. From Revision 100 to 472 the detected clone pairs vary between 68 and 88 before reaching 219 at Revision 473.
The number of clone pairs goes down to 122 at Revision 491 and decreases to 128 in the last revision. PRECINCT was able to detect 96.1\% (123/128) of the clone pairs that are detected by NICAD with a 100\% recall.
It took in average around 1 second for PRECINCT to execute on a Debian 8 system with Intel(R) Core(TM) i5-2400 CPU @ 3.10GHz, 8Gb of DDR3 memory.
It is also worth mentioning that the computer we used is equipped with SSD (Solid State Drive).
This impacts the running time as clone detection is a file intensive operation.
Finally, the PRECINCT was able to output 88.3\% less lines than NICAD.

JHotDraw starts with 196 clone pairs at Revision 1 and reaches a pick of 2048 at Revision 180. The number of clones  continues to go up until Revisions 685 and 686 where the number of pairs is 1229 before picking at 6538 and more from Revisions 687 to 721.
PRECINCT was able to detect 98.3\% of the clone pairs detected by NICAD (6490/6599) with 100\% recall while executing on average in 1.7 second (compared to 5.1 seconds for NICAD).
With JHotDraw, we can clearly see the advantages of incremental approaches.
Indeed, the execution time of PRECINCT is loosely impacted by the number of files inside the system as the blocks are constructed incrementally.
Also, we only compare the latest change to the remaining of the program and not all the blocks to each other as NICAD.
We also were able to reduce by 70.1\% the number of lines output by NICAD.

Finally, for Dnsjava, the number of clone pairs starts high with 258 clones and goes up until Revision 70 where it reaches 165. Another quick drop is observed at Revision 239 where we found only 25 clone pairs. The number of clone pairs stays stable until Revision 1030 where it reaches 273. PRECINCT was able to detect 82.8\% of the clone pairs detected by NICAD (226/273) with 100\% recall, while executing on average in 1.1 second while NICAD took 3 seconds in average. PRECINCT outputs  83.4\% less lines of code than NICAD.

Overall, PRECINCT prevented 97.7\% of the 7000 clones (in all systems) to reach the central source code repository while executing more than twice as fast as NICAD (1.2 seconds compared to 3 seconds in average) while reducing the output in terms of lines of text output the developers by 83.4\% in average. Note here that we have not evaluated the quality of the output of PRECINCT compared to NICAD's output. We need to conduct user studies for this. We are, however, confident, based on our own experience trying many clone detection tools, that a simpler and more interactive way to present the results of a clone detection tool is warranted. PRECINCT aims to do just that.

The difference in execution time between NICAD and PRECINCT stems from the fact that, unlike PRICINCT, NICAD is not an incremental approach. For each revision, NICAD has to extract all the code blocks and then compares all the pairs with each other. On the other hand, PRECINCT only extracts blocks when they are modified and only compares what has been modified with the rest of the program.

The difference in precision between NICAD and PRECINCT (2.3\%)  can be explained by the fact that sometimes developers commit code that does not compile.
Such commits will still count as a revision, but TXL fails to extract blocks that do not comply with the target language syntax.
While NICAD also fails in such a case, the disadvantage of PRECINCT comes from the fact that the failed block is saved and used as reference until it is changed by a correct one in another commit.
