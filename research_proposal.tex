\documentclass[11pt,a4paper, cuthesis]{report}
% \documentstyle{}


\usepackage{pgfgantt}

\usepackage{cuthesis}
\usepackage[english]{babel}
\usepackage{graphicx}
\usepackage{amsmath}%
\usepackage{multirow}
\usepackage{algorithm2e}
\usepackage{xargs}                      % Use more than one optional parameter in a new commands
\usepackage{xcolor}  % Coloured text etc.
\usepackage{hyperref}
\usepackage{listings}
\lstset{
    frame=single,
    breaklines=true,
    postbreak=\raisebox{0ex}[0ex][0ex]{\ensuremath{\color{red}\hookrightarrow\space}}
}

\usepackage{todonotes}

\author{Mathieu Nayrolles}
\title{Pragmatic Software Maintenace: \\ Tool and technics to support software maintenance at commit-time.}

\titleOfPhDAuthor{Mr.}          % or Ms., Mrs., Miss, etc. (only for PhD's)
\PhD                            % Masters by default
\dept{Electrical \& Computer Engineering}        %   default is Comp.Sci.
%\cosupervisor                   % if you also have a co-supervisor

%%%%%%%%%%%%%%%%%%%%%%%%%%%%%%%%%%%%%%%%%%%%%%%%%%%%%%%%%%%%%%%%%%%%%%%%%%%%%%%


\begin{document}


\begin{abstract}

  The maintenance and evolution of complex software systems account for more than 70\% software's life cycle.
  More than two decades of research have been conducted to improve our knowledge of these processes in terms of issue triaging, issue prediction, duplicate issue detection, issue reproduction and changes prediction.
  This research gave meaning to the millions of issues that can be found in project and revision management systems.
  Context-aware IDE and think tank in open source architecture, openned the path to approaches that support developers during their programming sessions by leveraging past knowledge and architectures.

  However, these techniques are still not broadly adopted by praticionners in large software companies.
  The main problem is the lack of actionable inteligence.
  Indeed, such systems tend to be seen as black boxes that yield false positive by end-users.
  Moreover, actions to resolve an identified problem can be hard to identify or to apply.

  In this research proposal, we present four approaches: (a) an online bug-fix search engine and API ({\tt BUMPER}, Bug MetarePository for dEveloper and Researcher), (b) an approach to automatically reproduce bug submitted to bug-report systems ({\tt JCHARMING}, Java CrasH Automatic Reproduction by directed Model checkING), (c) a recommendation system to propose the right auto-completion at the right time using actionnable inteligence ({\tt RESEMBLE}, REcommendation System based on cochangE Mining at Block LEvel) and, finally, (d) an approach to prevent bug insertion at commit-time by leveraging decade of open-source history ({\tt BIANCA}, Bug Insertion ANticipation by Clone Analysis at commit time).
  We also propose a taxonomy of bugs.
  When combined into {\tt pErICOPE} (Ecosystem Improve source COde during Programming session with real-time mining of common knowlEdge), these tools (i) provide the possibility to search related software artifacts using natural language, (ii) accurately reproduce field-crash in lab environment, (iii) recommend improvement or completion of block of code under edition and (iv) prevent the introduction of issues at commit time.

  This proposal develops these ideas while highlighting remaining issues and the PhD schedule.
\end{abstract}

\tableofcontents
\listoffigures
\listoftables

%%%%%%%%%%%%%%%%%%%%%%%%%%%%%%%%%%%%%%%%%%%%%%%%%%%%%%%%%%%%%%%%%%%%%%%%%%%%%%%
%% Body of Thesis goes here.
%%%%%%%%%%%%%%%%%%%%%%%%%%%%%%%%%%%%%%%%%%%%%%%%%%%%%%%%%%%%%%%%%%%%%%%%%%%%%%%

%!TEX root = research_proposal.tex

\pagenumbering{arabic}
\setcounter{page}{1}

\chapter{Introduction}

Maintenance activities are known to be costly and challenging \cite{Pressman2005}. Studies have shown that the cost of software maintenance can reach up to 70\% of the overall cost of the software development process \cite{HealthSocial2002}.

Lehman's laws of software evolution apply to three different types of software $S$, $P$ and $E$ \cite{lehman1980programs}. S-Software are written according to an exact specification of what that program can do, E-program are is written to perform some real-world activity; how it should behave is strongly linked to the environment in which it runs, and such a program needs to adapt to varying requirements and circumstances in that environment and, finally, P-software are program is written to implement certain procedures that completely determine what the program can do.

\begin{itemize}
	\item ``Continuing Change'' — an E-type system must be continually adapted or it becomes progressively less satisfactory
	\item ``Increasing Complexity'' — as an E-type system evolves, its complexity increases unless work is done to maintain or reduce it
	\item  ``Self Regulation'' — E-type system evolution processes are self-regulating with the distribution of product and process measures close to normal
	\item ``Conservation of Organisational Stability (invariant work rate)'' - the average effective global activity rate in an evolving E-type system is invariant over the product's lifetime
	\item ``Conservation of Familiarity'' — as an E-type system evolves, all associated with it, developers, sales personnel and users, for example, must maintain mastery of its content and behavior to achieve satisfactory evolution. Excessive growth diminishes that mastery. Hence the average incremental growth remains invariant as the system evolves.
	\item ``Continuing Growth'' — the functional content of an E-type system must be continually increased to maintain user satisfaction over its lifetime
	\item ``Declining Quality'' — the quality of an E-type system will appear to be declining unless it is rigorously maintained and adapted to operational environment changes
	\item ``Feedback System'' — E-type evolution processes constitute multi-level, multi-loop, multi-agent feedback systems and must be treated as such to achieve significant improvement over any reasonable base
\end{itemize}

On a lighter note, Brian Russell resumed this into what he --- ironically --- named the Laws of Software Relativity:

\begin{itemize}
\item As a software project approaches release, its mass increases.
\item The energy required to release a software project is inversely proportional to the time before a scheduled release.
\item It takes infinite energy to release a finished product on time; therefore, all software projects are both incomplete and late.
\item Time is relative to the observer of a software project. The last month of development appears to an outside observer to take a year.
\item If a software project becomes too large, it will collapse into a black hole. Time and money are absorbed, but nothing ever comes out.
\end{itemize}

While these laws, written from 1974 to 1996, still very much apply to nowadays software engineering, engineers and practitioners have created tools, techniques and processes to control their negative impacts.
For instance, most of real world project's---in opposition to test / very small size projects---host their source code on a version control system \cite{rochkind1975source} that is able is able to keep track of the different revisions of a system and of the different changes made by the developers of the team.
Version control systems help to control the {\it continuing change}, {\it increasing complexity} and {\it continuing growth} rules' effects.
In order to manage the {\it feedback systems} and {\it declining quality} rules, an organization uses issue \& project tracking system to assign tasks to developers, report unexpected behaviors or crashes and track advancement.

In the last decade, source code revision control system and issue \& tracking systems have grown to contain hundreds of thousands of revision, issues and tasks per project.
Naturally, this plethora of data pushed researchers across the world to conduct hundreds of studies in several active research fields: Bug reproduction, bug triaging, duplicated bug identification, bug comprehension, bug re-production.

Mining system and issue \& tracking and source code version control systems is perhaps one of the most active research fields today. The reason is that their analysis provides useful insight that can help with many maintenance activities such as bug fixing \cite{Weiss2007,Saha2014}, bug reproduction \cite{Artzi2008,Jin2012,Chen2013}, fault analysis \cite{Jiang2012,Jin2013}, etc. This increase of attention can be further justified by the emergence of many open source bug tracking systems, allowing open source software teams to make their bug reports available online to researchers.

\section{Preliminaries\label{sec:preliminaries}}


In this section, we explain what are version control systems (in \ref{sec:version-control}) and Issue \& project tracking system (in \ref{sec:issue-tracking}). If one were to be familiar with Svn\footnote{https://subversion.apache.org/}, Git\footnote{https://git-scm.com/}, Mercurial\footnote{https://mercurial.selenic.com/}, Bugzilla\footnote{https://www.bugzilla.org/}, Jira\footnote{https://www.atlassian.com/software/jira} and Github\footnote{https://github.com/}, one can skip to section \ref{sec:outline}.

\subsection{Version control systems\label{sec:version-control}}

Version control consists in maintaining the versions of files -- such as source code and other software artifacts.
This activity is extremely complex and cannot be done by hand on real world project\footnote{Once again, real world project qualifies projects that are done in an industrial environment rather than school or so.}.
Consequently, numerous software has been created to help practitioners manage the version of their software artifacts.
Each evolution of a software is a version\footnote{Software version is not to be confused with the version of a software which refer to the shipping of a final product to customers.} (or revision) and each version (revision) is linked to the one before through modifications of software artifacts.
These modifications, that consist in updating, adding or deleting software artifacts, can be referred as \texttt{diff}, {\tt patch} or {\tt commit}
\footnote{These names are not to be used interchangeably as difference exists.}.
Each \texttt{diff}, {\tt patch} or {\tt commit} have the following characteristics:

\begin{itemize}
\item Number of Files: The number of software artifacts that have been modified, added or deleted.
\item Number of Hunks: The number of consecutive blocks of modified, added or deleted lines in textual files. Hunks are useful to determine, in each file, how many different places the developer have modified.
\item Number of Churns:  The number of lines modified. However, the churn value for a line change should be at least two as the line had to be deleted first and then added back to the modification.
\end{itemize}

\subsubsection{Providers\label{sec:revision-provider}}

In this document, we will mainly refer to three version control systems: {\tt Svn}, {\tt Git} and, to a lesser extent, {\tt Mercurial}. {\tt SVN} is distributed by the Apache foundation and is a centralized concurrent version system that can handle conflict in the different versions of different developers and it is widely used.
At the opposite, {\tt Git} is a distributed revision control system --- originally developed by Linus Torvald --- where revisions can be kept locally for a while and then shared with the rest of the team.
Finally {\tt Mercurial} is also a distributed revision system, but share a lot of concepts with {\tt Svn}.
Consequently, it will be easier for folks used to {\tt Svn} to switch to a distributed revision system if they use {\tt Mercurial}.

\subsection{Issue \& Project Tracking Systems\label{sec:issue-tracking}}

Issue \& project tracking systems allows end-users to directly create bug reports (BRs) to report on system crashes and manager can create tasks to drive the evolution forward.
These systems are also used by development teams to manage the BRs, and keep track of the fixes.

The life cycle of an issue is as follows: After an issue is submitted by an end-user, it is set to the {\tt UNCONFIRMED} state until it receives enough votes or that a user with the proper permissions modifies its status to {\tt NEW}.
The bug is then assigned to a developer to fix it.
When the bug is in the {\tt ASSIGNED} state, the assigned developer(s) start working on the issue.
A fixed issue moves to the {\tt RESOLVED} state. Developers have five different possibilities to resolve an issue: {\tt FIXED}, {\tt DUPLICATE}, {\tt WONTFIX}, {\tt WORKSFORME} and {\tt INVALID}.

\begin{itemize}
	\item {\tt RESOLVED/FIXED}: A modification to the source code have been pushed, i.e., a changeset (also called a patch) has been committed to the source code management system and fixes the issue.
	\item {\tt RESOLVED/DUPLICATE}: A previously submitted issue is being processed. The bug is marked as duplicate of the original bug.
	\item {\tt RESOLVED/WONTFIX}: This is applied in the case where developers decide that a given bug will not be fixed.
	\item {\tt RESOLVED/WORKSFORME}: If the issue cannot be reproduced on the reported OS / hardware.
	\item {\tt RESOLVED/INVALID}: If the issue is not related to the software itself.
\end{itemize}

Finally, the issue is {\tt CLOSED} after it is resolved.
An issue can be reopened (sent to the {\tt REOPENED} state) and then assigned again if the initial fix was not adequate (the fix did not resolve the problem). The elapsed time between the issue marked as the new one and the resolved status are known as the {\it fixing time}, usually in days.
If the issue is reopened then the days between the time the issue is reopened and the time it is marked again as {\tt RESOLVED/FIXED} are cumulated. Issues can be reopened many times.

Tasks, follow a similar life cycle at the exception of the {\tt UNCONFIRMED} and {\tt RESOLVED} states.
Tasks are created by management and do not need to be confirmed in order to be {\tt OPEN} and {\tt ASSIGNED} to developers.
When a task is complete, it will not go to the {\tt RESOLVED} state, but to the {\tt IMPLEMENTED} state.
Issue are considered as problem to eradicate in the program and thus, fold into the maintenance activity.
Tasks are considered as new features or amelioration to include in the program and fold into the evolution activity.

Issue and tasks can (and must according to \cite{Bettenburg2008}) have a
severity. The severity is a classification of a issue to indicate the
degree of negative impact on the quality of software and can
evolve at any point during the lifecycle of the bug. The possible severities are blocker (blocks
development and/or testing work) critical (crashes, loss of
data, severe memory leak), major (major loss of function),
normal (regular issue, some loss of functionality under
specific circumstances), minor (minor loss of function, or
other problem where easy workaround is present) trivial
(cosmetic problem like misspelled words or misaligned text).


The relationship between an issue or task and the actual modification can be hard to establish and it has been a subject of various research studies (e.g., \cite{Antoniol2002,Bachmann2010,Wu2011}) for the simple reason that they are in two different systems: the revision and the issue management systems. While it is considered a good practice to link each issue with the source code revision system by indicating the issue $\#id$ on the modification message, more than half of the issue are not linked to a modification.


\subsubsection{Providers\label{sec:bug-provider}}

 In this study, we collect data from four different bug tracking systems: $Bugzilla$, $Jira$, $Github$ and $Sourceforge$. $Bugzilla$ belongs to the Mozilla foundation and have first been released in 1998. $Jira$ is provided by Altassian have been released 13 years ago, in 2002. $Bugzilla$ is 100\% open source and it's difficult to estimate how many project uses it.
 However, we can, without any risks envision that they have a great share of the market as the major organizations such as Mozilla, Eclipse and the Apache Software Foundation uses it.
 $Jira$, in the other hand, is a commercial software --- with a freemium business model --- and Altassian claims that they have 25,000 customers over the world.

 $Github$ and $Sourceforge$ are different from $Bugzilla$ and $Jira$ in a sense that they were created as revision system and evolve, later on, to add Issue and project management capabilities to their softwares. Nevertheless, this common particularity has the advantage to use the link between issues and source code.



\section{Motivation}

Architects, the ones that design buildings --- where mistakes cost lives --- spend at least five years at school and possibly their whole carriers to study, understand and reproduce great designs made by great architects.
Software architects, however, begin in programing 101 by displaying the famous ``{\tt Hello World}'' statement and exponentially increase the complexity of their programs over their years of study and work.
At some point, they will earn the title of software architect (or technical leader) because they have designed, maintained and evolved {\it enough} programs to be trustworthy on the matter.
However, unlike building architects they have to learn how to recognize, analyze and reproduce great architectural choices by themselves on the in addition of their day to day work.
Of course, software developers do learn good practices such as Design Patterns \cite{Gamma2008} but in a very few occasions they will be presented with a state-of-the-art program built by great developers (Amy Brown {\it et al.} propose exactly that in their books \cite{chansler2011architecture, AmyBrown2012, Armstrong2013}).

While this research is not about reforming how programming classes are taught, we still want to ease the access to this knowledge for developers during their programming sessions in order to ship better programs.

In this research proposal, we shift the focus from merely mining revision and issue management system, where knowledge of great developers lies, to integrate them in their rightful place: as the keystone of software development and evolution activities.
Extracting the ground truth from repositories helped engineers and practitioners to be better at building softwares as they know, for example,{\it how long it will take to fix a bug} \cite{Weiss2007}, {\it what's makes a good bug report} \cite{Bettenburg2008} or {\it how to fix long-lived bugs} \cite{Saha2014}.
Using these discoveries, tools can be created, on a per organization basis, to fit particular requirements such as programming languages, development processes or particular threshold. If we want to truthfully and deeply modify the software engineering landscape to have better softwares in terms of quality, maintainability and availability, we need to provide this information during the development, maintenance and evolution processes according to a specific context in an easy, reliable, free way.

If we look back at the history of software engineering, the increase of processors' speed allowed one to have a compiler on its own machine rather than sending one's code to the mainframe and receive compilation errors hours (days) later. This allowed, among other factors, the democratization of software engineering as {\it everyone}, belonging to a major organization or not, became able to build code. We believe that, it is now time to allow developers, engineering and practitioners, regardless of their programming language and contextual environment not only to write and build code but to write and built qualitative, robust, resilient, easy to maintain and to fix code. What better way to do so than to {\it stand on the shoulder of giants} by having access to the whole open sources repositories, including but not limited to, issues, tasks, bug fixes, patches, comments, good practices break down to the right level and provided at the right time during day to day programming sessions ?

What we concretely propose is an open-source, free, automatable tool suite that will allow everyone to (1) automatically reproduce field crashed in a controlled lab environment without any privacy concerns, (2) search in natural language the issues, comments, bugs, patches and fixes of tens of thousands of open source project, (3) prevent the insertion of defects in the source code during programming sessions by providing examples of the similar defect and how it has been fixed with a programming language abstraction. To support these tools we propose an empirical bug taxonomy (4).

\section{Thesis Contributions}


In this section we present the problems in the literature (Section \ref{sec:pb-litterature}), the research challenges we face in our work (Section \ref{sec:challenges}). Sections \ref{sec:scope} and \ref{sec:objective-thesis} present the scope and the contributions of this research.

\subsection{Problems in the Literature\label{sec:pb-litterature}}

\begin{itemize}
	\item {\bf Problem 1}: As shown by Figure \ref{fig:scholar}, the proportion of empirical studies and studies based on mining software repositories regarding to software quality has been increasing exponentially since 2010 (\cite{Kim2011a,Lee2011a,Sun2011,Bhattacharya2011,Tian2012a,Zimmermann2012, Shang2013, Chen2014, McIntosh, Hemmati2015} are some noticeable examples).
	While hundreds of bug prediction papers have been published by academia over the last decade, the developed tools and approaches fail to change developer behavior while deployed in industrial environment	\cite{Lewis2013}.
	This is mainly due to the lack of actionable message, i.e. messages that provide concrete steps to resolve the problem at hand.

	\begin{figure}[h!]
	  \centering
	  	    \includegraphics[scale=0.7]{media/scholar.png}
	    \caption{Proportion of papers containing ``Empirical Study'' or ``Mining software repository'' with regards to the paper in Software quality indexed by Google Scholar	\label{fig:scholar}}
	\end{figure}

	\item {\bf Problem 2}: The literature contains numerous papers about tools that improve the overall software quality with static \cite{Dangel2000, burn2003checkstyle, Hovemeyer2007, Moha2010} and dynamic \cite{Nayrolles,Nayrolles2013a,Palma2013} analysis. To the best of our knowledge approaches leveraging other sources to improve quality or efficiency mostly rely on web-search \cite{Brandt2009,Rahman2013,Montandon2013}.

	\item {\bf Problem 3}: There is no approach that supports the natural language search and comparison of issues, source code and tasks regardless of the project, repository, revision and issue management system and programming language. Such an approach could dramatically transform software engineering processes. Moreover, the data contained in these repositories lack a taxonomy, as for clone detection \cite{CoryKapser}, to classify the research.
\end{itemize}

\subsection{Research Challenges\label{sec:challenges}}

\begin{itemize}
	\item {\bf Challenge 1} : Issues \& projects and revision systems are plenty and they all have specific processes and limitations. Mining them all in order to have a representative model is challenging. Despite the parsing aspect, gigabytes of new data are generated every day thus, storing accessing to these data in reasonable time will require innovations in high density nosql databases \cite{Nayrolles2014b} and web servers \cite{Nayrolles2013b,Nayrolles2014c}. Moreover, creating the relationship between both systems is still an open issue as discussed in sections \ref{sec:issue-tracking} and \ref{rel:issue-rela}.

	\item {\bf Challenge 2}: Providing code samples break down to the right level, at the right time in order to solve a problem  or improve the current code in terms of quality, performances or reliability during a programming session will force us to improve current approaches of source code transformation and normalization \cite{Cordy2006a, Cordy2006, Roy2008, Cordy2011}.
\end{itemize}

\subsection{Scope of the research \label{sec:scope}}

The area of this research is to improve the processes of software engineering by providing contextual information, in order to improve the quality, the performance, the reliability of a given code during a programming session. These contextual information will come from the mining issue \& project and revision systems. Hence, we will not define what are the good or bad practices to improve the quality, the performance, the reliability of a given code but rely on the mined data.

\subsection{Thesis contributions\label{sec:objective-thesis}}

Figure \ref{fig:proposal} depicts our proposed solution for fulfilling the objectives we presented in section \ref{sec:objective-thesis}.
First of all, an end-user (team member) will report an issue (open a task) in one the organization issues \& project management system. This can be done in $Sourceforge$, $Bugzilla$, $JIRA$ or $Github$ which are the system we want to support first as described in section \ref{sec:bug-provider}.

\begin{figure}[h!]
	\centering
	\includegraphics[scale=0.9]{media/proposal.png}
	\caption{Proposed Architecture}
	\label{fig:proposal}
\end{figure}


Issues (tasks) are  mapped with their fixes (implementations) inside {\tt BUMPER} (BUg MetarePository for dEvelopers and Researchers). The source code is fetched from the supported version system: $Git$, $Svn$, $Mercurial$ presented in section \ref{sec:version-control}.
{\tt BUMPER} is a meta-repository that make issues, tasks and related source code searchable using natural language  (in opposition to structured query language).
When an issue is reported, {\tt JCHARMING} (Java CrasH Automatic Reproduction by directed Model checkING) will fetch the content of the issue and try to create a scenario to reproduce the on-field crash. In case of success, the developer assigned to this issue will be notified and the scenario stored in {\tt BUMPER}.
The developer assigned to the task or the issue will modify the code and, in real time, very much like intellisense or auto-completion in modern IDE, {\tt RESEMBLE} (REcommendation System based on cochangE Mining at Block LEvel) will propose improvements or follow up on the developer's code using the decades of history of {\tt BUMPER}. Once s/he is done, s/he submit a commit, patch or diff to the version system.
However, before s/he allowed to do so, {\tt BIANCA} (Bug Insertion ANticipation by Clone Analysis at commit time) will kick in and query {\tt BUMPER} looking for similar modifications in other projects and even other programming languages that led to the insertion of a defect in order to warn the user about potential hazardous code.


The bug taxonomy required to build {\tt BUMPER}, {\tt BUMPER} itself, {\tt JCHARMING}, {\tt RESSEMBLE} and {\tt BIANCA} are presented in sections \ref{sec:taxo}, \ref{sec:BUMPER}, \ref{sec:JCHARMING}, \ref{sec:RESEMBLE} and \ref{sec:BIANCA}, respectively. In addition, we list parts that have been published in peer-reviewed conferences in section \ref{sec:current-state} and our publication plan in \ref{sec:publication-plan}.

As a motivating example, we draft the following scenario. Table \ref{tab:bumper-hypo} presents hypothetical data stored in {\tt BUMPER} in terms of sequence \#id, sequence of code blocks, a flag to know if a said sequence introduced an issue in a given system and step to reproduce the issue if any.

\begin{table}[h!]
\centering
\begin{tabular}{c|c|c|c|c}
Seq \#ID & Language \#ID & Blocks & Root of Issue & Steps to reproduce \\ \hline \hline
1        & 1             & A-A-B-C-A-A   & Yes  & E-F-G         \\
2        & 1             & A-A-B-C       & No   & -         \\
3        & 2             & D-E-A-C       & No &  - \\ \hline \hline
\end{tabular}
\caption{Hypothetical {\tt BUMPER} data}
\label{tab:bumper-hypo}
\end{table}

During a programming session, let's assume that a developer has blocked $A-B-C$, then {\tt RESEMBLE} will recommends to transform the current code to $A-A-B-C$ as it seems to be the right thing to do.
If the developer follows {\tt RESEMBLE} recommendation and then adds another two $A$s and commit his/her changes.
The sequence is now $A-A-B-C-A-A$ and {\tt BIANCA} will raise a warning saying that this sequence is known to be the root of an issue and invite the developer to execute the steps $E-F-G$ -- that were produced by {\tt JCHARMING} -- in order to see if s/he did introduce a defect. Moreover, {\tt BIANCA} will take the time to compare $A-A-B-C-A-A$ and $D-E-A-C$ using our normalization algorithms even if they are not in the same programming language.
Finally, when a new issue is submitted, {\tt BUMPER} indexes it and {\tt JCHARMING} tries to reproduce it and update the step to reproduce part of {\tt BUMPER}.

We can envision the potential of such a system and the its complexity, knowing that it would contain millions of issues, hundreds of thousand projects, dozens of programming languages and will help developers leveraging the knowledge of other developers.

To summarize this thesis have four main contributions:

\begin{itemize}
	\item To provide a taxonomy of software issues to classify the research.
	\item To propose approaches to aggregate as many issues and revisions systems as possible.
	\item To propose approaches to reproduce field crashes in a lab environment using the issue content.
	\item To propose a context-aware IDE that will improve day-to-day programming session with concise and appropriate code samples.
\end{itemize}



\section{Outline\label{sec:outline}}

The remainder of this proposal is organized as follows. The following section discusses the problems we propose to solve and their motivations. Chapter \ref{chap:relwork} summarizes the related work.
Chapter \ref{chap:methodology} introduces the proposed approach, the framework, and the preliminary experiments along with the results.
Chapter \ref{chap:plan} presents the current state of the research, and highlights our key current contributions and future research directions along with with a possible publication schedule.
Finally, Chapter \ref{chap:conclusion} provide some concluding remarks and future works.

%!TEX root = research_proposal.tex

\chapter{Related Work\label{chap:relwork}}

\section{Crash reproduction}

In his Ph.D thesis \cite{Chen2013}, Chen proposed an approach named STAR (Stack Trace based Automatic crash Reproduction).
Using only the crash stack, STAR starts from the crash point and goes backward towards the entry point of the program.
During the backward process, STAR computes the required condition to reach the crash point using an SMT (Satisfiability Modulo Theories) solver named Yices \cite{Dutertre2006}.
The objects that satisfy the required conditions are generated and orchestrated inside a JUnit test case. The test is run and the resulting crash stack is compared to the original one. If both match, the bug is said to be reproduced. When applied to different systems, STAR achieved 60\% accuracy. \\

Jaygarl et al. \cite{Jaygarl} created OCAT (Object Capture based Automated Testing).
The authors’ approach starts by capturing objects created by the program when it runs on-field in order to provide them in an automated test process. Indeed the coverage of automated tests is often low due to the lack of correctly constructed objects.
Also, the objects can be mutated by means of evolutionary algorithms. These mutations target primitive fields in order to create even more objects and therefore improve the code coverage once more.
While not targeting the reproduction of a bug, OCAT is a well-known approach and was used as the main mechanism for bug reproduction. \\

Narayanasamy et al. \cite{Narayanasamy2005} proposed BugNet, a tool that continuously records program execution for deterministic replay debugging. According to the authors, the size of the recorded data needed to reproduce a bug with high accuracy is around 10MB. This recording is then sent to the developers and allows the deterministic replay of a bug. The authors argued that, with nowadays Internet bandwidth, the size of the recording is not an issue during the transmission of the recorded data, however, the instrumentation of the system is problematic since it slows down considerably the execution.\\

Jin et al. \cite{Jin2012} proposed BugRedux for reproducing field failures for in-house debugging. The tool aims to synthesize in-house executions that mimic field failures. To do so, the authors use several types of data collected in the field such as stack traces, crash stacks, and points of failure.
The data that successfully reproduced the field crash is sent to software developers to fix the bug. \\

Based on the success of BugRedux, the authors built F3 (Fault localization for Field Failures) \cite{Jin2013}.
F3 performs many executions of a program on top of BugRedux in order to cover different paths leading to the fault. It then generates many ‘pass’ and ‘fail’ paths which can lead to a better understanding of the bug. They also use grouping, profiling and filtering, to improve the fault localization process.\\

While being close to our approach, BugRedux and F3 may require the call sequence and/or the complete execution trace in order to achieve bug reproduction. When using only the crash traces (referred to as call stack at crash time in their paper), the success rate of BugRedux significantly drops to 37.5\% (6/16). The call sequence and the complete execution trace required to reach 100\% of bug reproduction can only be obtained through instrumentation and with an overhead ranging from 1\% to 1066\%.\\

Clause et al. \cite{Clause2007} proposed a technique for enabling and supporting debugging of field failures.
They record the execution of the program on the client side and propose to compress the generated data to the minimal required size to ensure that the reproduction is feasible.
This compression is also performed on the client side. Moreover, the authors keep traces of all accessed documents in the operating system and also compress/reduce them to the minimal.
Overall, they are able to reproduce on-field bug using a file weighting ≈70Kb. The minimal execution paths triggering the failure are then sent to the developers who can replay the execution on a sandbox, simulating the client’s environment. While efficient, this approach suffers from severe security and privacy issues. \\

RECORE (REconstructing CORE dumps) is a tool proposed by Rossler et al. \cite{Rossler2013}.
It instruments Java bytecode to wrap every method in a try and catch block while keeping a quasi-null overhead.
The tool starts from the core dump and tries (with evolutionary algorithms) to reproduce the same dump by executing the programs many times.
The set of inputs responsible for the failure is generated when the generated dump matches the collected one.
ReCrash \cite{Artzi2008} is a tool that aims to make software failures reproducible by preserving object states.
It uses an in-memory stack, which contains every argument and object clone of the real execution in order to reproduce a crash via the automatic generation of unit test cases.
Unit test cases are used to provide hints to the developers on the buggy code. This approach suffers from overhead when they record everything (between 13\% to 64\% in some cases).
The authors also propose an alternative in which they record only the methods surrounding the crash.
For this to work, the crash has to occur at least once so they could use the information causing the crash to identify the methods surrounding it when (and if) it appears. \\

JRapture \cite{Steven2000} is a capture/replay tool for observation-based testing.
The tool captures execution of Java programs to replay it in-house.  To capture the execution of a Java program, the authors used their own version of the Java Virtual Machine (JVM) and employ a lightweight, transparent capture process. Using their own JVM allows one to capture any interactions between a Java program and the system, including GUI, file, and console inputs, and on replay, it presents each thread with exactly the same input sequence it saw during capture.
Unfortunately, they have to make their customer use their own JVM in order to support their approach, which limits the generalization of the approach to mass-market software.\\

Finally, Zamfir et al. \cite{Parnin2011} proposed ESD, an execution synthesis approach which automatically synthesizes failure execution using only the stack trace information. However, this stack trace is extracted from the core dump and may not always contain the components that caused the crash.\\

Except for STAR, approaches targeting the reproduction of field crashes require the instrumentation of the code or the running platform in order to save the stack call or the objects to successfully reproduce bugs. As we discussed earlier, instrumentation can cause a massive overhead (1\% to 1066\%) while running the system. In addition, data generated at run-time using instrumentation may contain sensitive information.

\section{Issue and source code relationships\label{rel:issue-rela}}

Researchers have been studying the relationships between issues and source code repositories since more than two decades now.
To the best of our knowledge the first ones who conduct this type of study on a significant scale were Perry and Stieg \cite{PerryDewayneE.1993}.
In these two decades, many aspects of these relationships have been studied in length.
For example, researchers  interested themselves in ameliorating the issues report itself by specified guidelines to make good report \cite{Bettenburg2008} and try to further simplify the existing models \cite{Herraiz2008}.

Then, we can find approaches on how long it will take for a issues to get fixed \cite{Bhattacharya2011,Zhang2013,Saha2014} and where it should be fixed \cite{Zhou2012,Kim2013a}.
With the rapidly increasing number of issues, the community also interested itself in prioritizing the issues report compared to one another \cite{Kim2011c} and partially do so by predicting the severity of a issues \cite{Lamkanfi2010}.

Finally, researchers proposed approaches to predict which issues will get reopened \cite{Zimmermann2012,Lo2013} which issues report is a duplicate of which other one \cite{Jalbert2008,Bettenburg2008a,Tian2012a}.

Another field of study consist in assigning these issues reports, if possible automatically to the right developers through triaging  \cite{Anvik2006,Jeong2009,Tamrawi2011a,Bortis2013}
and which locations are likely to yield new bugs \cite{Kim2006,Kim2007}.

\section{Crash Prediction}

Predicting crash, fault and bug is very large and popular research area.
The main goal behind the pletor of papers is to save on manpower---being to more expensive resource to build software---by directing their efforts on locations likely to contain a bug, fault or crash.

There are two distinct trends in crash, fault and bug prediction in the papers accepted to major venues such as MSR, ICSE, ICSME and ASE:  analysis and current version analysis.

In the  analysis, researchers extract and interpret information from  system.
The idea being that the files or locations that are the most frequently changed are more likely to contain a bug.
Additionally, some of these approaches also assume that locations linked to a previous bug are likely to be linked to a bug in the future.

On the other hand, approaches using only the current version to predict bugs assume that the current version, i.e. its design, call graph, quality metrics and more, will trigger the appearance of the bug in the future.
Consequently, they do no require the history and only need the current source-code.

In the remaining of this section, we will describe approaches belonging to the two families.

\subsubsection{Change logs approaches}
\label{subs:Change logs approaches}

Change logs based approaches rely on mining the historical data of the application and more particularly, the source code \textit{diffs}.
A source code \textit{diffs} contains two versions of the same code in one file.
Indeed, it contains the lines of code that have been deleted and the one that have been added.
It is worth noting that, \textit{diffs} files do not represent the concept of modified line.
Indeed, a modified line will be represented by a deletion and an addition.
Researchers mainly use five metrics when dealing with \textit{diffs} files :

\begin{itemize}
  \item Number of files: The number of modified files in a given commit
  \item Insertions: The number of added lines
  \item Deletions: The number of deleted lines
  \item Churns: The number of deletion lines immediately followed by an insertion which give an approximation of how many lines have been modified
  \item Hunks: The number of consecutive blocks of lines. This gives an approximation of how many distinct locations have been edited to accomplish a unit of work.
\end{itemize}

Naggapan \textit{et al.} studied the churns metric and how it can be connected to the apparition of new defect in a complex software system.
They established that relative churns are, in fact, a better metric than classical churn \cite{Nagappan} while studying Windows Server 2003.

Hassan, interested himself with the entropy of change, i.e. how complex the change is \cite{Hassan2009}.
Then, the complexity of the change, or entropy, can be used to predict bugs.
The more complex a change is, the more likely it is to bring the defect with it.
Hassan used its entropy metric, with success, on six different systems.
Prior to this work, Hassan, in collaboration with Holt proposed an approach that highlights the top ten most susceptible locations to have a bug using heuristics based on \textit{diffs} file metrics \cite{Hassan2005}.
Moreover, their heuristics also leverage the data of the bug tracking system.
Indeed, they use the past defect location to predict new ones.
The conclusion of these two approaches has been that recently modified and fixed locations where the most defect-prone and comparison with frequently modified.

Similarly to Hassan and Hold,  Ostrand \textit{et al.} predict future crash location by combining the data from change and past defect location \cite{Ostrand2005}.
The main difference between Hassan and Hold and Ostrand \textit{et al.} is that Ostrand \textit{et al.} validate their approach on industrial systems as they are members of the AT\&T lab while Hassan and Hold validated their approach on open-source system.
This proved that these metrics are relevant for open-source and industrial system.


%
% he bug cache approach by Kim et al. uses
% the same properties of recent changes and defects as the
% top ten list approach, but further assumes that faults occur
% in bursts [13]. The bug-introducing changes are identified
% from the SCM logs. Seven open-source systems were used
% to validate the findings (Apache, PostgreSQL, Subversion,
% Mozilla, JEdit, Columba, and Eclipse). Bernstein et al. use
% bug and change information in non-linear prediction models
% [12]. Six eclipse plugins were used to validate the approach.
% Single-version approaches assume that the current design
% and behavior of the program influences the presence of
% future defects. These approaches do not require the history
% of the system, but analyze its current state in more detail,
% using a variety of metrics. One standard set of metrics used
% is the Chidamber and Kemerer (CK) metrics suite [17].
% Basili et al. used the CK metrics on eight mediumsized
% information management systems based on the same
% requirements [1]. Ohlsson et al. used several graph metrics
% including McCabe’s cyclomatic complexity on an Ericsson
% telecom system [2]. El Emam et al. used the CK metrics
% in conjunction with Briand’s coupling metrics [3] to predict
% faults on a commercial Java system [4]. Subramanyam et
% al. used CK metrics on a commercial C++/Java system [5];
% Gyimothy et al. performed a similar analysis on Mozilla
% [6]. Nagappan and Ball estimated the pre-release defect
% density of Windows Server 2003 with a static analysis tool
% [7]. Nagappan et al. used a catalog of source code metrics
% to predict post release defects at the module level on five
% Microsoft systems, and found that it was possible to build
% predictors for one individual project, but that no predictor
% would perform well on all the projects [8]. Zimmermann et
% al. applied a number of code metrics on Eclipse [18].
% Other Approaches. Zimmermann and Nagappan used
% dependencies between binaries in Windows server 2003
% to predict defect [19]. Marcus et al. used a cohesion
% measurement based on LSI for defect prediction on
% several C++ systems, including Mozilla [20]. Neuhaus
% et al. used a variety of features of Mozilla (past bugs,
% package imports, call structure) to detect vulnerabilities [21].
% Observations We observe that both case studies and the
% granularity of approaches vary. Varying case studies make
% a comparative evaluation of the results difficult. Validations
% performed on industrial systems are not reproducible, because
% it is not possible to obtain the data that was used. There
% is also some variation among open-source case studies, as
% some approaches have more restrictive requirements than
% others. With respect to the granularity of the approaches,
% some of them predict defects at the class level, others
% consider files, while others consider modules or directories
% (subsystems), or even binaries. While some approaches
% predict the presence or absence of bugs for each component,
% others predict the amount of bugs affecting each component
% in the future, producing a ranked list of components.
% These observations explain the lack of comparison between
% approaches and the occasional diverging results when
% comparisons are performed. In the following, we present a
% benchmark to establish a common ground for comparison.
%
%
% —While many bug prediction algorithms have been
% developed by academia, they’re often only tested and verified
% in the lab using automated means. We do not have a strong
% idea about whether such algorithms are useful to guide human
% developers. We deployed a bug prediction algorithm across
% Google, and found no identifiable change in developer behavior.
% Using our experience, we provide several characteristics that bug
% prediction algorithms need to meet in order to be accepted by
% human developers and truly change how developers evaluate their
% code.

%!TEX root = ../research_proposal.tex

\chapter{An Aggregate Bug Repository for Developers and Researchers\label{chap:bumper}}

In this chapter, we present {\tt BUMPER} (BUg Metarepository for dEvelopers and Researchers).
The role of {\tt BUMPER} is to aggregate information belonging to the version control and project management systems.
{\tt BUMPER} acts as our consolidated dataset.

The materials presented in this chapter are based on the following publications:

\begin{itemize}
	\item Nayrolles, M. \& Hamou-Lhadj, W. BUMPER: A Tool to Cope with Natural Language Search of Millions Bugs and Fixes. In Proceeding of the International Conference on Software Analysis, Evolution, and Reengineering (SANER'16) - Tool Track, pages 649-652, 2016.
	\item Nayrolles, M. \& Hamou-Lhadj, W. BUMPER: Bug Metarepository Search Engine for Developers and Researchers. Consortium for Software Engineering Research Fall, 2015.
\end{itemize}

%!TEX root = ../research_proposal.tex

With the goal to support the research towards analyzing relationships between bugs and their fixes we constructed a dataset of 380 projects, more than 100,000 resolved/fixed and with 60,000 changesets that were involved in fixing them from Netbeans and The Apache Software foundation's software that is (1) searchable in natural language at https://bumper-app.com, (2) contains clear relationships between the bug report and the code involved to fix it, (3) supports complex queries such as parent-child relationships, unions or disjunctions and (4) provide easy exports in json, csv and xml format.

In what follows, we will present the projects we selected. Then, we present the features related to the bugs and their fixes we integrate in BUMPER (BUg Metarepository for dEvelopers and Researchers) and how we construct our dataset. Then, we present the API, based on Apache Solr \cite{Nayrolles2014b}, which allows the NLP search with practical examples before providing research opportunities based on our dataset.


However, to the best of our knowledge, no attempt has been made towards building a unified and online dataset where all the information related to a bug, or a fix can be easily accessed by researchers and engineers.

\section{Data collection\label{sec:data-collection}}

Figure \ref{fig:bumper-approach} illustrates our data collection and analysis process that we present here and discuss in more detail in the following subsections. First, we extract the raw data from the two bug report management systems used in this study (Bugzilla\footnote{https://netbeans.org/bugzilla/} and Jira\footnote{https://issues.apache.org/jira/issues/?jql=}). The extracted data is consolidated in one database called BUMPER where we associate each bug report with its fix. The fixes are mined from different type of source versioning system. Indeed, Netbeans is based on mercurial\footnote{ http://mercurial.selenic.com/} while we used the git\footnote{http://git-scm.com/} mirrors\footnote{https://github.com/apache} for the Apache Foundation software.

\begin{figure}[h!]
  \centering
    \includegraphics{media/bumper-approach.png}
    \caption{Overview of the bumper database construction.
    \label{fig:bumper-approach}}
\end{figure}


In this study, we used two distinct datasets: Netbeans and
the Apache Software Foundation projects. Netbeans is an
integrated development environment (IDE) for developing
with many languages including Java, PHP, and C/C++. The
very first version of Netbeans, then known as Xelfi, appeared
in 1996. The Apache Software Foundation is a U.S non-profit
organization supporting Apache software projects such as the
popular Apache web server since 1999. The characteristics of
the Netbeans and Apache Software Foundation are presented in Table \ref{table:datasets}.

\begin{table}[h]
\begin{center}
\begin{tabular}{@{}c|c|c|c|c@{}}
\textbf{Dataset} & \textbf{R/F BR} & \textbf{CS} & \textbf{Files} & \textbf{Projects} \\ \hline \hline
Netbeans         & 53,258          & 122,632     & 30,595         & 39                \\
Apache           & 49,449          & 106,366     & 38,111         & 349               \\
Total            & 102,707         & 229,153     & 68,809         & 388               \\ \hline \hline

\end{tabular}
\end{center}

\caption{Datasets\label{table:datasets}}
\end{table}

Cumulatively, these datasets span from 2001 to 2014. In
summary, our consolidated dataset contains 102,707 bugs,
229,153 changesets, 68,809 files that have been modified to
fix the bugs, 462,848 comments, and 388 distinct systems.
We also collected 221 million lines of code modified to fix
the bugs, identified 3,284 sub-projects, and 17,984 unique
contributors to these bug report and source code version
management systems. Finally, the cumulated opening time for
all the bugs reaches 10,661 working years (3,891,618
working days).

We choose to use these two datasets because they exposed a great diversity in programming languages, teams, localization, utility and maturity. Moreover, the used different tools, i.e. Bugzilla, JIRA, Git and Mercurial, and therefore, BUMPER is ready to host any other datasets that used any composition of these tools.

\section{Architecture}

{\tt BUMPER} rely on a highly scalable architecture composed of two distinct servers as depicted in Figure \ref{fig:bumper-arch}. The first server, on the left, handles the web requests and runs three distinct components:

\begin{itemize}
	\item Pound is a lightweight open source reverse proxy program and application firewall.
	It is also served us to decode  to request to http. Translating an  request to http and then, use this HTTP request instead of the  one allow us to save the http's decryption time required at each step.
	Pound also acts as a load-balancing service for the lower levels.
	\item Translated requests are then handled to Varnish. Varnish is an HTTP accelerator designed for content-heavy and dynamic websites. What it does is caching request that come in and serve the answer from the cache is the cache is still valid.
	\item NginX (pronounced engine-x) is a web-server that has been developed with a particular focus on high concurrency, high performances and low memory usage.
\end{itemize}

On the second server, that concretely handles our data, we have the following items:

\begin{itemize}
	\item Pound. Once again, we use pound here, for the exact same reasons.
	\item SolrCloud is the scalable version of Apache Solr where the data can be separated into shards (e.g chunk of manageable size). Each shard can be hosted on a different server, but it's still indexed in a central repository. Hence, we can guarantee a low query time while exponentially increasing the data.
	\item Lucene is the full text search engine powering Solr. Each Solr server has its own embedded engine.
\end{itemize}

\begin{figure}[h!]
  \centering
    \includegraphics{media/bumper-arch.png}
    \caption{Overview of the bumper architecture.
    \label{fig:bumper-arch}}
\end{figure}

Request from users to the servers and the communication between our servers are going through the CloudFlare network.
CloudFlare acts as a content delivery network sitting between the users and the webserver.
They also provide an extra level of caching and security.

To give the reader a glimpse about the performances that this unusual architecture can yield; we are able to request and display the result of a specific request in less than 100 ms while our two servers are, in fact, two virtual machines sharing an AMD Opteron (tm) Processor 6386 SE (1 core @ 2,000 MHz) and 1 GB of RAM.

\section{UML Metamodel}

Figure \ref{fig:bumper-approach} presents the simplified {\tt BUMPER} metamodel that we designed according to our bug taxonomy presented in section \ref{fig:bug-taxo} and according to our future needs for {\tt JCHARMING}, {\tt RESSEMBLE} and {\tt BIANCA}.

\begin{figure}[h!]
  \centering
    \includegraphics{media/bumper-model.png}
    \caption{Overview of the bumper meta-model.
    \label{fig:bumper-approach} }
\end{figure}


An {\it issue} ({\it task}) is characterized by a {\it date}, {\it title}, {\it description}, and a {\it fixing time}. They are reported (created) by and assigned to {\it users}.
Also, {\it issues} ({\it tasks}) belong to {\it project} that are in {\it repository} and might be composed of {\it sub-projects}.
{\it Users} can modify an {\it issue} ({\it task})  during {\it life cycle events} which impact the {\it type}, the {\it resolution}, the {\it platform}, the {\it OS} and the {\it status}. {\it Issues} ({\it tasks}) are resolved (implemented) by {\it changeset} that are composed of {\it hunks}. {\it Hunks} contain the actual changes to a {\it file} at a given revision, which are versions of the {\it file} entity that belongs to a {\it Project}.


\section{Features}

In this section, we present the features of bug report and their fixes in details.

\subsection{Bug Report}

A bug report is characterized by the following features:


\begin{itemize}

\item ID: unique string id of the form bug\_dataset\_project\_bug\_id
\item Dataset: the dataset of which the bug is extracted from.
\item Type: The type help us to distinguish different type of entities in BUMPER, i.e the bugs, changesets and hunks. For bug report, the type is always set to BUG
\item Date: The date at which the bug report has been submitted.
\item Title: The title of the bug report.
\item Project: The project that this bug affects.
\item Sub\_project: The sub-project that this bug affects.
\item Full\_name\_project: The combination of the project and the sub-project.
\item Version: the version of the project that this bug affects
\item Impacted\_platform: the platform that this bug affects
\item Impacted\_os: the operating system that this bug affects
\item Bug\_status: The status of the bug. As in bumper, our main concern is on the relationship between of fix and a bug, we only have RESOLVED bugs
\item Resolution: How the bug was resolved. Once again, as we are interested in investigating the fixes and the bugs, we only have FIXED bugs.
\item Reporter\_pseudo: the pseudonym of the person who report the bug.
\item Reporter\_name: the name of the person who reported the bug
\item Assigned\_to\_pseudo: the pseudonym of the person who have \item been assigned to fix this bug
\item Assigned\_to\_name: the name of the person who have been assigned to fix this bug
\item Bug\_severity: the severity of a bug
\item Description: the description of the bug the reporter gave
\item Fixing\_time: The time it took to fix the bug, i.e the elapsed time between the creation of the BR and its modification to resolve/fixed, in minutes
\item Comment\_nb: How many comments have been posted on the bug report system for that bug
\item Comment: Contains one comment. A bug can have 0 or many comments
\item File: A file qualified name that has been modified in order to fix a bug. A bug can have 0 (in case we did not find its related commit) or many files.

\end{itemize}

We selected this set of features for bug report as they are the ones that are analyzed in many past and recent studies. In addition, bugs can contain 0 or many .

\subsection{Changesets}

In this section, we present the features that characterize changeset entities in BUMPER.

\begin{itemize}

\item ID: the SHA1 hash
\item User: the name and email of the person who submitted that commit
\item Date: the date at which this commit has been fixed
\item Summary: the commit message entered by the user
\item File: The fully qualified name of a file modified on that commits. A changeset can have 1 or many files.
\item Number\_files: How many files have been modified in that commit
\item Insertions: the number of inserted lines
\item Deletions: the number of deleted lines
\item Churns: the number of modified lines
\item Hunks: the number of sets of consecutive changed lines
\item Parent\_bug: the id of the bug this changeset belongs to.

\end{itemize}

In addition, changesets contain one or many hunks.

\subsection{Hunks}
A hunks are a set of consecutive lines changed in a file in order. A set of hunks form a fix that can be scattered across one or many files. Knowing how many hunks a fixed required and what are the changes in each of them is useful, as explained by [2] to understand how many places developers have to go to fix a bug.

Hunks are composed of:

\begin{itemize}
\item ID: unique id based on the files, the insertion and the SHA1 of the commits
\item Parent\_changeset: the SHA1 of the Changeset this hunk belongs to
\item Parent\_bug: the id of the bug this hunk belongs to.
\item Negative\_churns: how many lines have been removed in that hunk
\item Positive\_churns: how many lines have been added in that hunk
\item Insertion: the position in a file at which this hunk takes place.
\item Change: One line that have been added or removed. A Hunk can contain one or many changes.
\end{itemize}

\section{Application Program Interface (API)\label{sec:bumper-api}}

BUMPER is available for engineers and researchers at {\bf https://bumper-app.com} and take the form of a regular search engine. Bumper supports (1) natural language query, (2) parent-child relationships, query, (3) disjunctions and union between complex queries and (4) a straight forward export of query results in XML, CSV or JSON format.

Browsing BUMPER, the basic query mode, perform the following operation:

\begin{equation}
\begin{split}
(type:BUG~AND~report\_t:(``YOUR~TERMS''))~OR~(!parent~which=type``BUG'')~\\fix\_t:``YOUR~TERMS'')
\end{split}
\end{equation}

The first part of the query component of the query retrieves all the bugs that contains the $``YOUR~TERMS''$ query in at least one its features by selecting type: BUG and report\_t, which is an index composed of all the features of the bug, set to $``YOUR~TERMS''$.
Then, we merge this query with another one that reads \\
$(!parent~which=type``BUG'')fix\_t:~``YOUR~TERMS'')$.
In this one, we retrieve the parent documents, i.e the bugs, of fixes that contains $``YOUR~TERMS''$ in their $fix\_t$ index.
The $fix\_t$ index is, as for the BUG, an index based on all the fields of changeset and hunk both. As a result, we search seamlessly in the bug report and their fixes in natural language.

As a more practical example, Figure \ref{fig:bumper-live} illustrate a query on https://bumper-app.com. The search term is  ``{\it Exception}'' and we can see that 20,285 issues / tasks have been found in 25 ms This particular set of issues, displayed on the left side, match because they contain ``{\it Exception}'' in the issue report or in the source code modified to fix this issue (implement this task). Then on the right side of the screen, the selected issue (task) is displayed. We can see the basic characteristic of the issue (task) followed by comments and finally, the source code.

\begin{figure}[h!]
  \centering
    \includegraphics[scale=0.3]{media/bumper-live.png}
    \caption{Screenshot of https://bumper-app.com with ``Exception'' as research.
    \label{fig:bumper-live}}
\end{figure}


Moreover, BUMPER supports AND, OR, NOR operators and provide results in order of seconds.

As we said before, BUMPER is based on Apache Solr which have an incredibly rich API that is available online\footnote{ http://lucene.apache.org/solr/resources.html}.

{\tt BUMPER} serves as data repositories for the upcoming approaches presented in the chapters.


\chapter{JCHARMING: Java CrasH Automatic Reproduction by directed Model checkING\label{chap:jcharming}}

In this chapter, we present {\tt JCHARMING} (Java CrasH Automatic Reproduction by directed Model checkING).
{\tt JCHARMING} is an approach to reproduce field-crash.
Every time a bug report is submitted to a project management system and aggregated by {\tt BUMPER}, {\tt JCHARMING} will try to reproduce it.
In case of success, the steps to reproduce the bug are saved in {\tt BUMPER}.

The materials presented in this chapter are based on the following publications:

\begin{itemize}
	\item Nayrolles, M. , Hamou-Lhadj, W., Tahar, S. & Larsson, A. (2016). A Bug Reproduction Approach Based on Directed Model Checking and Crash Traces. Journal of Software: Evolution and Process. Wiley. 2016. (Accepted).
	\item Nayrolles, M. , Hamou-Lhadj, W., Tahar, S. & Larsson, A. JCHARMING : A Bug Reproduction Approach Using Crash Traces and Directed Model Checking. In Proceeding of the International Conference on Software Analysis, Evolution, and Reengineering (SANER'15), pages 101-110, 2015. (Best Paper Award).
\end{itemize}

\input{methodology/methodology_jcharming}

\chapter{Preventing Clone Insertion\label{chap:clone-detection-pragmatic}}

The adoption of tools and processes that aim to support human maintainers during maintenance is relatively low \cite{Lewis2013,Foss2015,Layman2007,Ayewah2007,Ayewah2008,Johnson2013,Norman2013, Lopez2011}.
Human maintainers agree that such tools are beneficial, but, in addition of disrupting their workflow, these tools tend to have a high false positive rate.
Also, the way in which the warnings are presented, among other things, are barriers to use\cite{Johnson2013}.
When asked for their opinions, human maintainers agree to the following characteristics for maintenance-oriented tools\cite{Hovemeyer2004, Lopez2011, Lewis2013}:

\begin{itemize}
	\item Actionable messages. Presenting a warning about bug-proneness of a given line is not enough.
	Clear actions to improve the source code should be provided
	\item Obvious reasoning. The conditions that led to a given warning should be understandable by the maintainer.
	If the conditions are hidden in complex statistical models, then maintainers cannot review them and will find it difficult to trust the tool.
	\item Scaling. Industrial sized project contains thousand files and dependencies which can be updated many times a day.
	Maintenance-oriented tools should not hinder the productivity of maintainers.
	\item Contextualization. Warnings and messages should always be contextualized with respect to the project at hand and not generic rules.
	\item Integration. Developers and maintainers are overwhelmed by the amount of existing tools.
	Yet, their daily use three different kinds of tools.
	An integrated development environment, a versioning system and a project tracking system to produce, version and manage their software, respectively.
	Maintenance-oriented tools should fit in the existing ecosystem rather than complexifying the deployment process.
\end{itemize}

In this chapter, we present {\tt PRECCINT} (PREventing Clones INsertion at Commit Time) and {\tt RESEMBLE} (REcommendation System based on cochangE Mining at Block LEvel) that have the characteristics described by human maintainer.

\input{methodology/methodology_precinct}

\\
Using the block extracted by {\tt PRECINCT}, we can recommend actions to the developer.
To do so, we plan to build {\tt RESEMBLE}.
{\tt RESEMBLE} is presented in the next section.


%!TEX root = ../research_proposal.tex

\subsection{RESEMBLE - REcommendation System based on cochangE Mining at Block LEvel\label{sec:RESEMBLE}}

{\tt RESEMBLE} (REcommendation System based on cochangE Mining at Block LEvel) is a contextual recommendation system that will take the form of another pre-commit hook. {\tt RESEMBLE} will leverage the branches' history and the blocks extracted by {\tt PRECINCT} to identify, on the fly, sub-optimum or hazardous code and display and actual solution to the problem.


\subsubsection{Mining sequences of changes}

In the data mining field, ARM is a
well-established method for discovering co-occurrences between attributes
in the objects of a large data set \cite{Gregory1991,HEIKKI1997}. Plain
associations have the form $X \rightarrow Y$, where $X$ and $Y$,
called the \textit{antecedent} and the \textit{consequent}, respectively, are sets of descriptors
(purchases by a customer, network alarms, or any other general kind of events).
Even though plain association rules could serve some relevant information, we are interested here in
the sequences of changes that we believe will yield more precise result. Indeed, we think that similar modifications are often done in the same order by the same developer (e.g top to bottom or bottom to top).
We, therefore, adopt a variant called sequential association rules in which
both $X$ and $Y$ become sequences of descriptors.
Moreover, our sequences follow a temporal order with the antecedent preceding the consequent.
Rules of this type mined from changes reveal crucial
information about the likelihood of blocks of code to be modified together in a programming session
and, more importantly, in a specific order.
For instance, a strong rule \emph{$Block_A$ $,$ $Block_B$} implies \emph{$Block_C$} would mean that after modifying $Block_A$ and then $Block_B$, there are good chances that the developer needs to modify $Block_C$.
The conciseness of this example should not confuse the reader as in practical cases
the sequences appearing in a rule can be of an arbitrary length.
Furthermore, the strength of the rule is measured by the \textit{confidence} metric.
In probabilistic terms,
it measures the conditional probability of C appearing down the line.
Beside that, the significance of a rule, i.e. how many times it appears in the data, is provided by its \textit{support} measure.
To ensure only rules of potentially high interestingness are mined,
the mining task is tuned by minimal thresholds to
output only the sufficiently high scores for both metrics.

To extract the association rules from changes,
two choices were possible. On one hand, sequential pattern mining and rule mining algorithms
have been designed for structures that are slightly more general than the ones used here.
In fact, sequential patterns are defined on transactions that represent sequences of sets.
Efficient sequential pattern miners have been published, e.g. the PrefixSpan method \cite{Pei2004}.
On the other hand, sequence of changes do not compile to fully-blown sequential transactions as the underlying structures are mere sequences of individual elements. Such data has been known since at least the mid-90s but received less attention by the data mining community, arguably because it is less challenging to mine.
In the general data mining literature, mining from pure sequences, as opposed to sequences made of sets, has been addressed under the name of episode mining \cite{HEIKKI1997}.
Episodes are made of \textit{events} and in a sense, code changes are events. Arguably the largest body of knowledge on the subject belongs to the web usage mining field: The input data is again a system log, yet this time the log of requests sent to a web server \cite{Pei2000}.
It is noteworthy that sequential patterns are more general than the pure sequence ones, hence mining algorithms designed for the former might prove to be less efficient when applied to the latter (as additional steps might be required for listing all significant set).
Nevertheless, to jump-start our experimental study, we used a sequential pattern/rule miner that has the advantage to be freely available on the web\footnote{http://www.philippe-fournier-viger.com/spmf/}.
Although it has not been optimized for pure sequences its performances are more than satisfactory.

\subsubsection{Code Normalization\label{sec:resemble-normalization}}

Normalizing code is the action of making its structure consistent throughout the program according to a defined model. In our case, we only normalize the block of code that are in the current sequential association rules. To do so, we improve and combine several technologies such as source code transformation \cite{Cordy2006,Cordy2006a}, source code pretty-printing \cite{Roy2008}, flexible source code normalization \cite{Cordy2011}.

More specifically, the code first goes to a pretty printer. A pretty-printer is a component that will slightly transform the code in order to obtain consistent control structure.
Concretely, spaces will be added, accolade moved and tabulation added for an $if$, a $while$ and others structures to always appear the same way, regardless of the programming language.
Then, the code is normalized several times. Each normalization is different and targets specific feature of the code.
For example, we have one normalization that removes completely the variable names and replace the types by the highest known object in the object oriented hierarchy before {\tt Object} itself\footnote{For Java programs}. Another normalization only keeps the structures of the source code by normalizing both variables names and types.

\subsubsection{Comparing normalization\label{sec:resemble-comparing}}

Comparing different normalizations is the easiest step and can be done efficiently using using the longest common sequence (LCS) algorithm \cite{hirschberg1977algorithms}.
If the LCS is above an user-defined threshold, then a two different behaviors can be observed:

\begin{itemize}
	\item If the modified blocks' --- and potentially the ones that are likely to be modified after according to our sequential association rules --- normalizations match the normalizations of blocks of code that have been removed in past history. {\tt RESEMBLE} recommends the replacing code as a better solution.
  In other words, if blocks $A$, $B$ and $C$ have been replaced by blocks $A'$, $B'$ and $C'$ and blocks $D$, $E$ and $F$ normalizations match $A$, $B$ and $C$, then, {\tt RESEMBLE} recommends to transform $D$, $E$ and $F$ to look like $A'$, $B'$ and $C'$. This could lead to the introduction of software clones but we argue that (i) software clones are not always harmfull \cite{Juergens2009} and (ii) informing the developer that $A'$, $B'$ and $C'$ exist could lead him to re-use these blocks.
	\item If the modified blocks' --- and potentially of the ones that are likely to be modified after according to our sequential association rules --- normalizations match the normalization of blocks of code that are present in the history. {\tt RESEMBLE} computes the differences between the developer code and the history code in order to suggests what the developer have to do next.
\end{itemize}

\subsubsection{Planned experiments}

We did not started the experiments for {\tt RESEMBLE} yet. Nevertheless, we plan to conduct the following experiments:

\begin{itemize}
	\item Full history test with Normalization 1.
	\item Full history test with Normalization 2.
	\item An human study where:
	\begin{itemize}
		\item Developers use {\tt RESEMBLE} in order to determine whether or not developers take into account our recommendations to avoid inserting defects in the code.
		\item Developers use {\tt RESEMBLE} in order to determine whether or not developers take into account our recommendations to complete their modification according to what we found in the history.
		\item Rate the suggested solution in a scale from 1 to 10 in order to determine if the proposed change pattern does resolve the current problem.
	\end{itemize}
\end{itemize}

We believe that {\tt RESEMBLE} will be a real asset in a developer tool belt in order to ship better code in terms of quality, performances and security.
However, as {\tt RESEMBLE} aims to provide recommendations in at commit-time, using the local history and ressources, it will not be able to be as exhaustive as an offline process.
To fill this gap, we built {\tt BIANCA} that we present in the next section.


In summary, using {\tt PRECINCT} and {\tt RESSEMBLE} we are able to provide contextualized and accurate information in a non-intrusive manner.

\chapter{Preventing Bug Insertion Using Clone Detection}

%!TEX root = ../research_proposal.tex


{\tt BIANCA} (Bug Insertion ANticipation by Clone Analysis at merge time) is an approach that we propose and which aims to prevent the insertion of bugs at commit-time. Many tools exist to prevent a developer to ship {\it bad} code \cite{Dangel2000,Hovemeyer2007,Moha2010} or to identify {\it bad} code after executions (e.g in test or production environment) \cite{Nayrolles,Nayrolles2013a}. However, these tools rely on metrics and rules to statically and/or dynamically identify sub-optimum code. {\tt BIANCA} is different than the approaches presented in the previous sections because it mines and analyses the change patterns in commits and matches them against past commits known to have introduced a defect in the code (or that have just been replaced by better implementation).
Also, {\tt BIANCA} is an offline approach that is triggered by a merge request.
When  maintainers see that their work are ready to be integrated with the main branch, they open a merge request\footnote{Also known as pull request}.
Merging a task branch is not an instantaneous process as the code need to pass code review.
{\tt BIANCA} leverages this {\it down} time to perform a complete history check on all projects contained in {\tt BUMPER}.

Figure \ref{fig:bianca-approach} presents an overview of our approach.

\begin{figure}[h!]
  \centering
    \includegraphics{media/bianca-approach.png}
    \caption{The BIANCA Approach
    \label{fig:bianca-approach}}
\end{figure}

{\tt BIANCA} builds a model where each issue is represented by three versions of the same file.
These three versions are stored in {\tt BUMPER}.
The first version $n$ is called the {\it stable state} because the code of this version was used to fix an issue.
The $n-1$ version, however, is called the {\it unstable state} as it was marked as containing an issue.
Finally, the third version is called the {\it before state} and represents the file before the introduction of the bug.
Hereafter, we refer to the {\it before state} as $n-2$.
{\tt BIANCA} extracts the change patterns form $n-2$ to $n-1$ and from $n-1$ to $n$.
It aslo generates the changes to go from $n-2$ to $n$.

When a developer commits new modifications, {\tt BIANCA} extracts the change pattern from the version $n_{dev}$ (current version) and $n-1_{dev}$ (version before modification) of the developer's source code and compares this change patterns to known $n-2$ to $n-1$ patterns.
If $n_{dev}$ to $n-1_{dev}$ matches a $n-2$ to $n-1$ then it means that the developer is inserting a known defect in the source code.
In such a case, {\tt BIANCA} will propose the related $n-1$ to $n$ patterns to the developer, so s/he could improve the source code and will show the related $n-2$ for the $n$ pattern so the developer will learn how to s/he should have modified the code in the first place.

Moreover, if the issue was previously reproduced by {\tt JCHARMING}, then {\tt BIANCA} will display the steps to reproduce it.


To extract the change patterns and compare them, we use the same technique as the one presented in section \ref{sub:Extract and Save Blocks}.
The third and the fourth normalizations are removing all {\it less} important calls in the normalization one and two of {\tt RESEMBLE}.
We classify a call as less-important if, for example, it only does display-related functionalities such as generating HTML or printing something to the console. Finally, the fourth normalization will transform the code to an intermediate language of our own that will allow us to compare source code implemented in different programming languages.

Then, as in {\tt RESEMBLE}, if the LCS is above a user-defined threshold, then a warning is raised by {\tt BIANCA} alerting the developer that the commit is suspected to insert a defect.
The given defect is shown to the developer and can either force the commit if s/he don't find the warning relevant or abort the commit.

We believe that the warning, alongside the previously mined change patterns and steps to reproduce the suspected default --- provided by {\tt JCHARMING}, if available --- will statisfy developers in terms of actionable inteligence.
Thus, {\tt BIANCA} could succeed, where other tools failed, at being used in industrial environment	\cite{Lewis2013}.


\section{Early experiments}

We have assessed the efficiency of {\tt BIANCA} with the same datasets we used to build our bug taxonomy proposed in Section \ref{chap:taxonomy}.

\begin{table}[h]
\begin{center}
\begin{tabular}{@{}c|c|c|c|c@{}}
\textbf{Dataset} & \textbf{Fixed Issues} & \textbf{Commit} & \textbf{Files} & \textbf{Projects} \\ \hline \hline
Netbeans         & 53,258          & 122,632     & 30,595         & 39                \\
Apache           & 49,449          & 106,366     & 38,111         & 349               \\
Total            & 102,707         & 229,153     & 68,809         & 388               \\ \hline \hline

\end{tabular}
\end{center}

\caption{Datasets\label{table:datasets-bianca}}
\end{table}


We ran two different experiments using the two first normalizations we described in Section \ref{sec:bianca-approach}. Both experiments consider only a few months of history, from April to August 2008. While this could hinder the accuracy of our results, these five-month  history data contain 167,597 commits related to bug fixes. We believe  that the results to be representative.

The first experiment yields the result presented in Figure \ref{fig:bianca-exp-1}.

\begin{figure}[h!]
  \centering
    \includegraphics[scale=0.55]{media/bianca-13.png}
    \caption{{\tt BIANCA} warnings from April to August 2008 using the first normalization.
    \label{fig:bianca-exp-1}}
\end{figure}

With the first normalization, {\tt BIANCA} raised 69,519 warnings out of 167,597 (41.5\%) analysed commits.
Out of these 69,519, 13.4\% turned out to be false positives. A false positive is a healthy commit that has been tagged as introducing a bug by {\tt BIANCA}.
However, false positives have to be dealt with carefully in this study as the commit might have introduced a bug but the bug has not been reported yet.

In our second experiment, we used the second normalization and {\tt BIANCA} raised 83,627 warnings out of 167,597 (48.89\%) commit we analyze. However, the false positive rate increased to 21\%. Figure \ref{fig:bianca-exp-2} shows the results.

\begin{figure}[h!]
  \centering
    \includegraphics[scale=0.55]{media/bianca-20.png}
    \caption{{\tt BIANCA} warnings from April to August 2008 using the second normalization.
    \label{fig:bianca-exp-2}}
\end{figure}

{\tt BIANCA} experiments are still in their early stages and we are still trying to improve our normalization in order to reduce the false positive rate.


\chapter{A taxonomy to classify the research\label{chap:taxonomy}}

%!TEX root = ../research_proposal.tex

In order to classify the research on the different fields related to software maintenance, we can reason about types of bugs at different levels. For
example, we can group bugs based on the developers that fix
them or using information about the bugs such as crash traces.


Our aim is not to improve testing as it is the case in the work of Eldh \cite{Eldh2001} and Hamill et al.\cite{Hamill2014}.
Our objective is to propose a classification that can allow researchers in the filed of mining bug 9 repositiories to use the taxonomy as a new criterion in triaging, prediction, and reproduction of bugs.
By analogy, we can look at the proposed bug taxonomy in a similar way as the clone taxonomy presented by Kapser and Godfrey \cite{CoryKapser}.
The authors proposed seven types of source code clones and then conducted a case study, using their classification, on the file system module of the Linux operating system.
This clone taxonomy continues to be used by researchers to build better approaches for detecting a given clone type and being able to effectively compare approaches with each other.

In this section, we are interested in bugs that share similar fixes.
By a fix, we mean a modification (adding or deleting lines of
code) to an exiting file that is used to solve the bug. With this
in mind, the relationship between bugs and fixes can be
modeled using the UML diagram in Figure \ref{fig:bug-taxo-diag}. The diagram
only includes bugs that are fixed. From this figure, we can
think of four instances of this diagram, which we refer to as
bug taxonomy or simply bug types (see Figure \ref{fig:bug-taxo}).



\begin{figure}[h!]
  \centering
    \includegraphics[scale=0.5]{media/bug-taxo-class-diag.png}
    \caption{Class diagram showing the relationship between bugs and fixed
    \label{fig:bug-taxo-diag}}
\end{figure}

\begin{figure}[h!]
  \centering
    \includegraphics[scale=0.8]{media/bug-taxo.png}
    \caption{Proposed Taxonomy of Bugs
    \label{fig:bug-taxo}}
\end{figure}


The first and second types are the ones we intuitively know
about. Type 1 refers to a bug being fixed in one single location
(i.e., one file), while Type 2 refers to bugs being fixed in more
than one location. In Figure 2, only two locations are shown
for the sake of clarity, but many more locations could be
involved in the fix of a bug. Type 3 refers to multiple bugs that
are fixed in the exact same location. Type 4 is an extension of
Type 3, where multiple bugs are resolved by modifying the
same set of locations. Note that Type 3 and Type 4 bugs are
not duplicates, they may occur when different features of the
system fail due to the same root causes (faults).
We conjecture that knowing the proportions of each type
of bugs in a system may provide insight into the quality of the
system. Knowing, for example, that in a given system the
proportion of Type 2 and 4 bugs is high may be an indication
of poor system quality since many fixes are needed to address
these bugs. In addition, the existence of a high number of
Types 3 and 4 bugs calls for techniques that can effectively
find bug reports related to an incoming bug during triaging.
This is similar to the many studies that exist on detection of
duplicates (e.g., \cite{Runeson2007,Sun2010,Nguyen2012}), except that we are not looking for
duplicates but for related bugs (bugs that are due to failures of
different features of the system, caused by the same faults). To
our knowledge, there is no study that exmpirically examines
bug data with these types in mind, which is the main objective
of this section. More particularly, we are interested in the
following research questions:

\begin{itemize}
	\item RQ1: What are the proportions of different types of bugs?
	\item RQ2: How complex is each type of bugs?
	\item RQ3: How fast are these types of bugs fixed?
\end{itemize}


\section{Study Setup}

Figure \ref{fig:bug-taxo-flow} illustrates our data collection and analysis
process that we present here and discuss in more detail in the
following subsections. First, we extract the raw data from the
two bug report management systems used in this study
(Bugzilla and Jira). Second, we extract the fix to the bugs
from the source code version control system of Netbeans and
Apache (Maven and Git).

\begin{figure}[h!]
  \centering
    \includegraphics{media/bug-taxo-flow.png}
    \caption{Data collection and analysis process of the study
    \label{fig:bug-taxo-flow}}
\end{figure}


The extracted data is
consolidated in one database where we associate each bug
report to its fix. We mine relevant characteristics of BRs and
their fixes such as opening time, number of comments,
number of times the BR is reopened, number of changesets
for BR and the number of files changed and lines modified
for fixes or patch. Finally, we analyze these characteristics to
answer the aforementioned research questions (RQ).


\section{Study Design}

We describe the design of our study by first stating the
research questions, and then explaining the variables, and
analysis methods we used to answer these questions. We
formulate three research questions (RQs) with the ultimate
goal to improve our understanding of each bug type. We
focus, however, on Types 2 and 4. This is because these bugs
require multiple fixes. They are therefore expected to be more
complex.
The objective of the first research question is to analyze
the proportion of each type of bugs. The remaining two
questions address the complexity of the bugs and the bug
fixing duration according to the type of bugs.
1)

\subsection{RQ 1: What are the proportions of different types of bugs?}

The answer to this question provides insight into the
distribution of bugs according to their type with a focus on
Type 2 and 4 bugs. As discussed ealier, knowing, for
example, that bugs of Type 2 and 4 are the most predominant
ones suggests that we need to investigate techniques to help
detect whether an incoming bug is of Types 2 and 4 by
examining historical data. Similarly, if we can automatically
identify a bug that is related to another one that has been fixed
then we can reuse the results of reproducing the first bug in
reproducing the second one.

{\bf Hypothesis}: To answer this question, we analyze whether
Type 2 and 4 bugs are predominant in the studied systems, by
testing the null hypothesis:

\begin{itemize}
	\item $H_{01A}$ : The proportion of Types 2 and 4 does not
change significantly across the studied systems
\end{itemize}


We test this hypothesis by observing both a “global”
(across systems) and a “local” predominance (per system) of
the different types of bugs. We must observe these two
aspects to ensure that the predominance of a particular type of
bug is not circumstantial (in few given systems only) but is
also not due to some other, unknown factors (in all systems
but not in a particular system).

{\bf Variables}: We use as variables the amount of resolved/fixed
bugs of each type (1, 2, 3 and 4) that are linked to a fix
(commit). As mentioned earlier, duplicate bugs are excluded.
These are marked as resolved/duplicate in our dataset.

{\bf Analysis Method}: We answer RQ1 in two steps. The first
step is to use descriptive statistics; we compute the ratio of
Types 2 and 4 bugs and the ratio of Types 1 and 3 bugs to the
total number of bugs in the dataset. This shows the
importance of Types 2 and 4 bugs compared to Types 1 and 3
bugs.

In the second step, we compare the proportions of the
different types of bugs with respect to the system where the
bugs were found. We build the contingency table with these
two qualitative variables (the type and studied system) and
test the null hypothesis H 01A to assess whether the proportion
of a particular type of bugs is related to a specific system or
not.

We use the Pearson's chi-squared test to reject the null
hypothesis $H_{01A}$ . Pearson’s chi-squared independence test is
used to analyze the relationship between two qualitative data,
in our study the type bugs and the studied system. The results
of Pearson’s chi-squared independence test are considered
statistically significant at $\alpha$ = 0.05. If p-value $\le$ 0.05, we
reject the null hypothesis $H_{01A}$ and conclude that the
proportion of types 3 and 4 bugs is different from the
proportion of type 1 and 2 bugs for each system.

\subsection{RQ 2: How complex is each type of bugs?}

We address the relation between Types 2 and 4 bugs and
the complexity of the bugs in terms of severity, duplicate and
reopened.We analyze whether Types 2 and 4 bugs are more
complex to handle than Types 1 and 3 bugs, by testing the
null hypotheses:

\begin{itemize}
 \item  $H_{02S}$ : The severity of Types 2 and 4 bugs is not
significantly different from the severity of Types 1 and 3
 \item  $H_{02D}$ : Types 2 and 4 bugs are not significantly more
likely to get duplicated than Types 1 and 3.
 \item  $H_{02R}$ : Type 2 and 4 bugs are not significantly more
likely to get reopened than Types 1 and 3.
\end{itemize}

{\bf Variables}: We use as independent variables for the
hypotheses $H_{02S}$ , $H_{02D}$ , $H_{02R}$ the bug type (if the bug is from
Types 2 and 4 or if it is from Types 1 and 3). For $H_{02S}$  we use
the severity as dependent variable to assess the relationship
between the bug severity and the bug type. For $H_{02D}$
(respectively $H_{02R}$) we use a dummy variable duplicated
(reopened) to assess if a bug has been duplicated (reopened)
at least once or not. This will be used to assess the
relationship between the type of the bugs and the fact that the
bug is more likely to be reopened or duplicated.

{\bf Analysis Method}: For each hypothesis, we build a
contingency table with the qualitative variables type of bugs
(2 and 4 or 1and 3) and the dependent variable duplicated
(respectively reopened) and the severity variable.

We use the Pearson’s chi-squared test to reject the null
hypothesis $H_{02D}$ (respectively $H_{02R}$ ) and $H_{02S}$. The results of
Pearson’s chi-squared independence test are considered
statistically significant at $\alpha$ = 0.05. If a p-value $\le$ 0.05, we
reject the null hypothesis  $H_{02D}$ (respectively $H_{02R}$) and
conclude the fact that the bug is more likely to be duplicated
(respectively reopened) is related to the type of the bug and
we reject $H_{02S}$ and conclude that the severity level of the bug
is related to the bug type.

\subsection{RQ 3 : How fast are these types of bugs fixed ?}

In this question, we study the relation between the
different types of bugs and the fixing time. We are interested
in evaluating whether developers take more time to fix Types
2 and 4 bugs than Type 1 and 3, by testing the null hypothesis:

\begin{itemize}
	\item $H_{03}$ : There is no statistically-significant difference
between the duration of fixing periods for Types 2 and
4 bugs and that of Types 1 and 3 bugs.
\end{itemize}


{\bf Variables}: To compare the bug fixing time with respect to
their type, we use as independent variable the type Ti of a
bug Bi, to distinguish between Types 1 and 3 bugs and Types
2 and 4 bugs. We consider as dependent variable the fixing
time, FTi, of the bug Bi. We compute the fixing time FTi of a
bug Bi. The fixing time FTi is the time between when the bug
is submitted to when it is closed/fixed.

{\bf Analysis Method}: We compute the (non-parametric) Mann-
Whitney test to compare the BR fixing time with respect to
the BR type and analyze whether the difference in the average
fixing time is statistically significant. We use the Mann-
Whitney test because, as a non-parametric test, it does not
make any assumption on the underlying distributions. We
analyze the results of the test to assess the null hypothesis
$H_{03}$ . The result is considered as statistically significant at $\alpha$ =
0.05. Therefore, if p-value $\le$ 0.05, we reject the null
hypothesis H 03 and conclude that the average fixing time of
Types 1 and 3 bugs is significantly different from the average
fixing time of Types 2 and 4 bugs.

\section{Study result and discussion}

In this section, we report on the results of the analyses we
performed to answer our research questions. We then dedicate
a section to discussing the results.

\subsection{RQ 1 : What are the proportions of different types of
bugs?}

Figure \ref{fig:bug-taxo-rq1} shows the percentage of the different types of
bugs. As shown in the figure, we found that 65\% of the bugs
are from Types 2 and 4. This shows the predominance of this
type of bugs in all the studied systems. Figure 5 shows the
repartition per dataset. We can see that Netbeans and Apache
have 66\% and 64\% bugs of Type 1and 3, respectively. To
ensure that this observation is not related to a particular
system, we perform Pearson’s chi-squared test across the
studied systems. Table \ref{tab:bug-taxo-rq1} shows the contingency table for the
studied systems and the result of Pearson’s chi-squared test.
The results show that there is statistically significant
difference between the proportions of the different types of
bugs.

\begin{table}[h!]
\centering
\begin{tabular}{c|c|c|c}
{System} & {Type 1 and 3} & {Type 2 and 4} & {Pearson’s chisquared p-value}        \\  \hline \hline
Apache       & 4910               & 8626               & \multirow{2}{*}{p-value \textless 0,0001} \\ \cline{1-3}
Netbeans     & 9050               & 17586              & \\ \hline \hline
\end{tabular}
\caption{Contingency table and Pearson's chi-squared tests\label{tab:bug-taxo-rq1}}
\end{table}

\begin{figure}[h!]
  \centering
    \includegraphics[scale=0.7]{media/bug-taxo-rq1.png}
    \caption{Proportions of different types of bugs
    \label{fig:bug-taxo-rq1}}
\end{figure}

Table \ref{tab:bug-taxo-rq1-prop} shows the number of bugs for each type of bugs
and the percentage of each type of bugs. We can see that
Types 3 and 4 bugs represent 28.33\% and 61.21\% of the total
of bugs, respectively. Types 1 and 2 represent only 6.78\% and
3.74\%. Together, Types 3 and 4 bugs represent almost 90\%
of the total number of bugs linked to a commit.

\begin{table}[h!]
\centering

\begin{tabular}{c|c|c|c|c|c}
Datasets                  & T1        & T2       & T3        & T4        & Total                  \\ \hline \hline
\multirow{2}{*}{Netbeans} & 776       & 240      & 8372      & 17366     & \multirow{2}{*}{26754} \\
                          & (2.90\%)  & (0.90\%) & (31.29\%) & (64.91\%) &                        \\ \hline
Apache                    & 1968      & 1248     & 3101      & 7422      & \multirow{2}{*}{13739} \\
                          & (14.32\%) & (9.08\%) & (22.57\%) & (54.02\%) &                        \\ \hline
\multirow{2}{*}{Total}    & 2744      & 1488     & 11473     & 24788     & \multirow{2}{*}{40493} \\
                          & (6.78\%)  & (3.74\%) & (28.33\%) & (61.21\%) & \\ \hline \hline
\end{tabular}
\caption{Proportion of bug types in amount and percentage}
\label{tab:bug-taxo-rq1-prop}
\end{table}


\noindent\fbox{%
    \parbox{\textwidth}{%
      We can thus reject the null hypothesis $H_{01A}$ and
conclude that there is a predominance of Types 2 and
4 bugs in all studied systems and this observation is
not related to a specific system.
    }%
}

\subsection{RQ 2 : How complex is each type of bugs?}

Figure \ref{fig:bug-taxo-rq2-prop-apache} and \ref{fig:bug-taxo-rq2-prop-netbeans} show the proportion of each bug type with
respect to their severity for each dataset. Table V shows the
proportion of each bug type with repect to their severity and
dataset. For Netbeans, the bugs we examined in our dataset
are either labeled as Blocker or Normal (despite the fact that
Netbeans uses Bugzilla that supports all the severity levels
presented in the previous section).

\begin{figure}[h!]
  \centering
    \includegraphics[scale=0.6]{media/bug-taxo-rq2-prop-apache.png}
    \caption{Proportions of Types 1 and 3 versus Types 2 and 4 with respct to their severity in the Apache dataset.
    \label{fig:bug-taxo-rq2-prop-apache}}
\end{figure}

\begin{figure}[h!]
  \centering
    \includegraphics[scale=0.6]{media/bug-taxo-rq2-prop-netbeans.png}
    \caption{Proportions of Types 1 and 3 versus Types 2 and 4 with respct to their severity in the Netbeans dataset.
    \label{fig:bug-taxo-rq2-prop-netbeans}}
\end{figure}

For the Apache dataset, the severity levels range from
Blocker to Trivial as shown in Figure \ref{fig:bug-taxo-rq2-prop-apache}. Figure \ref{fig:bug-taxo-rq2-prop-netbeans} shows that
in Netbeans around 67\% of Types 2 and 4 bugs are normal.
The same holds for Types 1 and 3 bugs (66\% are considered
of normal severity). This indicates that most Types 2 and 4
bugs and Types 1 and 3 bugs are not critical in the Netbeans
dataset. For the Apache dataset, the results indicate that the
majority of the bugs are considered of major severity (66\%
for Types 1 and 3 and 72\% for Types 2 and 4). It is
challenging to understand the discrepancy between the two
datasets partly because of the way the severity is assigned to
BRs.

Table \ref{tab:bug-taxo-rq2-chi} shows the result of the Pearson chi-squared tests
for the $H_{02S}$, $H_{02D}$ and $H_{02R}$ hypotheses.

\begin{table}[h!]
\centering
\begin{tabular}{c|c|c}
System                    & Factor     & \begin{tabular}[c]{@{}c@{}}Pearson’s chisquared\\ p-value\end{tabular} \\ \hline \hline
\multirow{3}{*}{Apache}   & Severity   & p-value \textless 0.005                                                \\
                          & Reopened   & p-value \textless 0.005                                                \\
                          & Duplicated & p-value \textless 0.005                                                \\ \hline \hline
\multirow{3}{*}{Netbeans} & Severity   & p-value \textless 0.005                                                \\
                          & Reopened   & p-value \textless 0.005                                                \\
                          & Duplicated & p-value \textless 0.005  \\  \hline \hline
\end{tabular}

\caption{Pearson's chi squared p-values for the severity, the reopen and the duplicate factor with respect to a dataset}
\label{tab:bug-taxo-rq2-chi}
\end{table}

\noindent\fbox{%
    \parbox{\textwidth}{%
      According to the results of the test (p-value < 0.005),
we reject the null hypothesis $H_{02S}$ and conclude that
there is a significant difference between the severity of
Types 1 and 3 bugs and the severity of Types 2 and 4
bugs.
    }%
}

\begin{table}[h!]
\centering
\begin{tabular}{c|c|c|c|c}
Severity                  & T1      & T2      & T3      & T4      \\
\hline \hline
\multicolumn{5}{c}{Netbeans}                                      \\ \hline
\multirow{2}{*}{Blocker}  & 340     & 109     & 2850    & 5687    \\
                          & 43.81\% & 45.42\% & 34.04\% & 32.75\% \\ \hline
\multirow{2}{*}{Normal}   & 436     & 131     & 5522    & 11678   \\
                          & 56.19\% & 54.58\% & 65.96\% & 67.25\% \\
                          \hline
\multirow{2}{*}{Total}    & 776     & 240     & 8372    & 17365   \\
                          & 100\%   & 100\%   & 100\%   & 100\%   \\
\hline \hline
\multicolumn{5}{c}{Apache}                                        \\ \hline
\multirow{2}{*}{Blocker}  & 68      & 53      & 115     & 329     \\
                          & 3.46\%  & 4.25\%  & 3.71\%  & 4.43\%  \\
                          \hline
\multirow{2}{*}{Critical} & 84      & 44      & 213     & 565     \\
                          & 4.27\%  & 3.53\%  & 6.87\%  & 7.61\%  \\
                          \hline
\multirow{2}{*}{Major}    & 1245    & 811     & 2096    & 5427    \\
                          & 63.26\% & 64.98\% & 67.59\% & 73.12\% \\
                          \hline
\multirow{2}{*}{Minor}    & 408     & 276     & 501     & 899     \\
                          & 20.73\% & 22.12\% & 16.16\% & 12.11\% \\
                          \hline
\multirow{2}{*}{Trivial}  & 113     & 31      & 159     & 161     \\
                          & 5.74\%  & 2.48\%  & 5.13\%  & 2.17\%  \\
                          \hline
\multirow{2}{*}{Total}    & 1918    & 1215    & 3084    & 7381    \\
                          & 100\%   & 100\%   & 100\%   & 100\%  \\
\hline \hline
\end{tabular}
\caption{Proportion of each bug type with respect to severity.}
\label{tab:bug-taxo-rq2-severity}
\end{table}

Table \ref{tab:bug-taxo-rq2-dup} shows the occurrences of duplicate and reopened
bugs with respect to their bug type in each dataset. In
Netbeans, the proportion of Type 1 bugs that are marked as
source of duplicate is 6.06\%, 4.59\% for Type 2 bugs, 5.09\%
for Type 3 bugs and 5.87\% for Type 4 bugs with a total of
1503 bugs over 26754 (5.62\%). In Apache, the proportion of
Type 1 bug marked a source of a duplicate is 2.59\% and
2.24\%, 1.61\% and 2.91\% for Types 2, 3 and 4, respectively.

\begin{table}[h!]
\centering
\begin{tabular}{c|c|c|c|c|c}
Type                   & T1     & T2     & T3     & T4     & Total  \\ \hline \hline
\multicolumn{6}{c}{Netbeans}                                      \\ \hline \hline
\multirow{2}{*}{Dup.}  & 6.06\% & 4.59\% & 5.09\% & 5.87\% & 5.62\% \\
                       & (47)   & (11)   & (426)  & (1019) & (1503) \\ \hline
\multirow{2}{*}{Reop.} & 4.38\% & 7.08\% & 4.81\% & 7.09\% & 6.30\% \\
                       & (34)   & (17)   & (403)  & (1231) & (1685) \\ \hline
\multicolumn{6}{c}{Apache}                                        \\ \hline \hline
\multirow{2}{*}{Dup}   & 2.59\% & 2.24\% & 1.61\% & 2.91\% & 2.51\% \\ \
                       & (51)   & (28)   & (50)   & (216)  & (345)  \\ \hline
\multirow{2}{*}{Reop}  & 5.59\% & 6.49\% & 3.10\% & 6.90\% & 5.82\% \\
                       & (110)  & (81)   & (96)   & (512)  & (799)  \\ \hline
\multicolumn{6}{c}{Total}                                         \\ \hline \hline
\multirow{2}{*}{Dup}   & 3.57\% & 2.62\% & 4.15\% & 4.98\% & 4.56\% \\
                       & (98)   & (39)   & (476)  & (1235) & (1848) \\ \hline
\multirow{2}{*}{Reop}  & 5.25\% & 6.59\% & 4.35\% & 7.03\% & 6.13\% \\
                       & (144)  & (98)   & (499)  & (1743) & (2484) \\ \hline
\end{tabular}
\caption{Percentage and occurences of bugs duplicated by other bufs and reopenned with respect to their bug type and dataset.}
\label{tab:bug-taxo-rq2-dup}
\end{table}

Second, we analyze the reopened bugs to see the link
between the reopening and the type of bugs. We perform
Pearson’s chi-squared test to reject the null hypothesis $H_{02R}$.

\noindent\fbox{%
    \parbox{\textwidth}{%
      According to the results of the test (p-value <
0.005), we reject the null hypothesis $H_{02R}$ and
conclude that there is a significant relationship
between the reopening of a bug and its type.
    }%
}

Third, we analyze the duplicated bugs to see if there is a
link between the bug type and the fact duplication. We
perform Pearson’s chi-squared test to reject the null
hypothesis $H_{02D}$.

\noindent\fbox{%
    \parbox{\textwidth}{%
      According to the results of the test (p-value <
0.005), we reject the null hypothesis $H_{02D}$ and
conclude that there is a significant relationship
between the duplication of a bug and its type.
    }%
}

\subsection{RQ 3 : How fast are these types of bugs fixed ?}

Figure \ref{fig:bug-taxo-rq3} shows the fixing time for Types 1 and 3 versus
Types 2 and 4 for Netbeans and the Apache Software
Foundation. In Netbeans, 98.96 and 137.05 days are required
to fix Types 1 and 3 and Types 2 and 4, respectively. In
Apache, 55.76 and 85.48 days are required to fix Types 1 and
3 and Types 2 and 4, respectively.


\begin{figure}[h!]
  \centering
    \includegraphics[scale=0.8]{media/bug-taxo-rq3.png}
    \caption{Fixing time of Types 1 and 3 versus fixing time of Types 2 and 4.
    \label{fig:bug-taxo-rq3}}
\end{figure}

Table \ref{tab:bug-taxo-rq3} shows the average fixing time of bugs with respect
to their bug type in each dataset.

\begin{table}[h!]
\centering
\begin{tabular}{c|c|c|c|c|c}
Dataset  & T1    & T2     & T3     & T4     & Average \\ \hline \hline
Netbeans & 97.66 & 117.42 & 100.26 & 156.67 & 118.00  \\ \hline
Apache   & 73.48 & 118.12 & 38.04  & 52.83  & 70.62   \\ \hline
Total    & 85.57 & 117.77 & 69.15  & 104.75 & 94.31  \\ \hline \hline
\end{tabular}
\caption{Average fixing time with respect to bug type
    \label{tab:bug-taxo-rq3}}
\end{table}

We analyze the difference in the fixing time of bugs with
respect to their bug type by conducting a Mann-Whitney test
to assess $H03$.The results show that the difference between
the fixing time of Types 2 and 4 and Types 1 and 3 is
statistically significant (p-value < 0,005).

\noindent\fbox{%
    \parbox{\textwidth}{%
      Therefore, we can reject the null hypothesis $H03$ and
conclude that the fixing of Types 2 and 4 bugs takes
more time than the fixing of Types 1 and 3 bugs.
    }%
}

\subsubsection{Dicussion}

{\bf Repartition of bug types}: One important finding of this
study is that there is significantly more Types 2 and 4 bugs
than Types 1 and 3 in all studied systems. Moreover, this
observation is not system-specific. The traditional one-bug/
one-fault way of thinking about bugs only accounts for 35\%
of the bugs. We believe that, recent triaging algorithms
\cite{Jalbert2008,Jeong2009,Khomh2011a,Tamrawi2011a} can benefit from these findings by developing
techniques that can detect Type 2 and 4 bugs. This would
result in better performance in terms of reducing the cost,
time and efforts required by the developers in the bug fixing
process.

{\bf Severity of bugs}: We discussed the severity and the
complexity of a bug in terms of its likelihood to be reopened
or marked as duplicate (RQ2). Although clear guidelines exist
on how to assign the severity of a bug, it remains a manual
process done by the bug reporter. In addition, previous
studies, notably those by Khomh et al. \cite{Khomh2011a}, showed that severity is not a consistent/trustworthy characteristic of a BR,
which lead to he emergence of studies for predicting the
severity of bugs (e.g., \cite{Lamkanfi2010,Lamkanfi2011,Tian2012}). Nevertheless, we
discovered that there is a significant difference between the
severities of Types 1 and 3 compared to Types 2 and 4.

{\bf Complexity of bugs}: At the complexity level, we use the
number of times a bug is reopened as a measure of
complexity. Indeed, if a developer is confident enough in
his/her fix to close the bug and that the bug gets reopened it
means that the developer missed some dependencies of the
said bug or did not foresee the consequences of the fix.
We found that there is a significant relationship between
the number of reopenings and type of a bug. In other words,
there is a significant relationship between the complexity and
the type of a given bug. In our datasets, Types 1 and 3 bugs
are reopened in 1.88\% of the cases, while Types 2 and 4 are
reopened in 5.73\%. Assuming that the reopening is a
representative metric for the complexity of bug, Types 2 and
4 are three times more complex than Types 1 and 3. Finally, if
we consider multiple reopenings, Types 2 and 4 account for
almost 80\% of the bugs that reopened more than once and
more than 96\% of the bug opened more than twice.
While current approaches aiming to predict which bug
will be reopen use the amount of modified files \cite{Shihab2010,Zimmermann2012,Lo2013}, we
believe that they can be improved by taking into account the type of a the bug. For example, if we can detect that an
incoming bug if of Type 2 or 4 then it is more likely to
reopened than a bug of Type 1 or 3. Similarly, approaches
aiming to predict the files in which a given bug should be
fixed could be categorized and improved by knowking the
bug type in advance \cite{Zhou2012,Kim2013a}.

{\bf Impact of a bug}: To measure the impact of bugs in end-users
and developers, we use the number of times a bug is
duplicated. This is because if a bug has many duplicates, it
means that a large number of users have experienced and a
large number of developers are blocked the failure.
We found that there is a significant relationship between
the bug type and the fact that it gets duplicated. Types 1 and 3
bugs are duplicated in 1.41\% of the cases while Types 2 and 4
are duplicated in 3.14\%. Assuming that the amount of
duplication is an accurate metric for the impact of bug, Types
2 and 4 have more than two times bigger impact than Types 1
and 3. Similarly to reopening, if we consider multiple
duplication, Types 2 and 4 account for 75\% of the bugs that
get duplicated more than once and more than 80\% of the bugs
that get duplicated more than twice.
We believe that approaches targeting the identification of
duplicates \cite{Bettenburg2008a,Jalbert2008,Sun2010,Tian2012a}  could leverage this taxonomy to
achieve even better performances in terms of recall and
precision.

{\bf Fixing time}: Our third research question aimed to determine
if the type of a bug impacts its fixing time. Not only we found
that the type of a bug does significantly impact its fixing time,
but we also found that, in average Types 2 and 4, stay open
111.26 days while Types 1 and 3 last for 77.36 days. Types 2
and 4 are 1.4 time longer to fix than Types 1 and 3.We
therefore believe that, approaches aiming to predict the fixing
time of a bug (e.g., \cite{Panjer2007,Bhattacharya2011,Zhang2013}) can highly benefit from
accurately predicting the type of a bug and thereforebetter
plan the required man-power to fix the bug.
In summary, Types 2 and 4 account for 65\% of the bugs
and they are more complex, have a bigger impact and take
longer to be fixed than Types 1 and 3 while being equivalent
in terms of severity.

Our taxonomy aimed to analyse: (1) the
proportion of each type of bugs; (2) the complexity of each
type in terms of severity, reopening and duplication; (3) the
required time to fix a bug depending on its type. The key
findings are:
\begin{itemize}
  \item Types 2 and 4 account for 65\% of the bugs.
  \item Types 2 and 4 have a similar severity compared to
Types 1 and 3.
  \item Types 2 and 4 are more complex (reopening) and have
a bigger impact (duplicate) than Types 1 and 3.
  \item It takes more time to fix Types 2 and 4 than Types 1
and 3.
\end{itemize}

Our taxonomy and results can be built upon in order to classify
past and new researches in several active areas such as bug
reproduction and triaging, prediction of reopening,
duplication, severity, files to fix and fixing time. Moreover, if
one could predict the type of a bug at submission time, all
these areas could be improved.



\chapter{Reflection on pragmatic software maintenance}

Architects, the ones that design buildings --- where mistakes cost lives --- spend at least five years at school and possibly their whole carriers to study, understand and reproduce great designs made by great architects.
Software architects, however, begin in programing 101 by displaying the famous ``{\tt Hello World}'' statement and exponentially increase the complexity of their programs over their years of study and work.
At some point, they will earn the title of software architect (or technical leader) because they have designed, maintained and evolved {\it enough} programs to be trustworthy on the matter.
However, unlike building architects, they have to learn how to recognize, analyze and reproduce great architectural choices by themselves in addition of their day to day work.
Of course, software developers do learn good practices such as design patterns \cite{Gamma2008} but in a very few occasions they will be presented with a state-of-the-art program built by great developers (Amy Brown {\it et al.} propose exactly that in their books \cite{chansler2011architecture, AmyBrown2012, Armstrong2013}).

While our research is not about reforming how programming classes are taught, we still want to ease the access to this knowledge for developers during their programming sessions in order to ship better programs.

We shift the focus from merely mining revision and issue management system, where knowledge of great developers lies, to integrate them in their rightful place: as the keystone of software development and evolution activities.
Extracting the ground truth from repositories helped engineers and practitioners to be better at building softwares as they know, for example, {\it how long it will take to fix a bug} \cite{Weiss2007}, {\it what makes a good bug report} \cite{Bettenburg2008} or {\it how to fix long-lived bugs} \cite{Saha2014}.
Using these discoveries, tools can be created, on a per organization basis, to fit particular requirements such as programming languages, development processes or particular threshold. If we want to truthfully and deeply modify the software engineering landscape to have better softwares in terms of quality, maintainability and availability, we need to provide this information during the development, maintenance and evolution processes according to a specific context in an easy, reliable, actionable, free way.

If we look back at the history of software engineering, the increase of processors' speed and decrease of their price allowed one to have a compiler on its own machine rather than sending one's code to the mainframe and receive compilation errors hours (days) later. This allowed, among other factors, the democratization of software engineering as {\it everyone}, belonging to a major organization or not, became able to build code. We believe that, it is now time to allow developers, engineering and practitioners, regardless of their programming language and contextual environment, not only to write and build code but to write and built qualitative, robust, resilient, easy to maintain and to fix code. What better way to do so than to {\it stand on the shoulder of giants} by having access to all the open sources repositories, including but not limited to, issues, tasks, bug fixes, patches, comments, good practices break down to the right level and provided at the right time during day to day programming sessions?

We believe that the {\it right} times for distilling this information are when a developer version its code.

%!TEX root = research_proposal.tex

\chapter{Remaining Work to Complete the Thesis\label{chap:plan}}

In this chapter, we summarize the state of the research and the work that needs to be completed in order to finalize the thesis.
We proceed according to the contributions listed on Section \ref{sec:objective-thesis}.


\section{An aggregate bug  repository for  developers  and  Researchers}

We introduced {\tt BUMPER} (BUg Metarepository for  dEvelopers  and  Researchers),  a  web-based  infrastructure
that  can  be  used  by  software  developers  and  researchers  to access  data  from  diverse  repositories  using  natural  language queries in a transparent manner, regardless of where the data was originally created and hosted.
{\tt BUMPER} have been showcased in the following publications:

\begin{itemize}
	\item Nayrolles, M. \& Hamou-Lhadj, W. BUMPER: A Tool to Cope with Natural Language Search of Millions Bugs and Fixes. In Proceeding of the International Conference on Software Analysis, Evolution, and Reengineering (SANER'16) - Tool Track, pages 649-652, 2016.
	\item Nayrolles, M. \& Hamou-Lhadj, W. BUMPER: Bug Metarepository Search Engine for Developers and Researchers. Consortium for Software Engineering Research Fall, 2015.
\end{itemize}

We consider this contribution to be 100\% complete.

\section{A bug reproduction technique based on a combination of crash traces and model checking.}

In this work, we proposed an approach, called {\tt JCHARMING} (Java CrasH Automatic Reproduction by directed Model checkING) that uses a combination of crash traces and model checking to automatically reproduce bugs that caused field failures.
{\tt JCHARMING} have been showcased in the following publications:

\begin{itemize}
	\item Nayrolles, M. , Hamou-Lhadj, W., Tahar, S. & Larsson, A. (2016). A Bug Reproduction Approach Based on Directed Model Checking and Crash Traces. Journal of Software: Evolution and Process. Wiley. 2016. (Accepted).
	\item Nayrolles, M. , Hamou-Lhadj, W., Tahar, S. & Larsson, A. JCHARMING : A Bug Reproduction Approach Using Crash Traces and Directed Model Checking. In Proceeding of the International Conference on Software Analysis, Evolution, and Reengineering (SANER'15), pages 101-110, 2015. (Best Paper Award).
\end{itemize}

We consider this contribution to be 100\% complete.

\section{An incremental approach for preventing bug and clone insertion at commit time}

We presented {\tt PRECCINT} (PREventing Clones INsertion at Commit Time), {\tt RESEMBLE} (REcommendation System based on cochangE Mining at Block LEvel) and {\tt BIANCA} (Bug Insertion ANticipation by Clone Analysis at merge time) in chapters \ref{chap:clone-detection-pragmatic} and \ref{chap:bianca}.

The efficiency of {\tt PRECCINT} have been accessed.
Early experiments have been conducted for {\tt BIANCA}.
However, we need to conduct additional experiments to measure the effiency {\tt RESEMBLE} and {\tt BIANCA}.

We consider this contribution to be 40\% complete.

\section{A new taxonomy of bugs based on the location of the correction --- an empirical Study}

We investigated the relationship between bugs by examining their locations of the fixes in chapter \ref{chap:taxonomy}.
We still need to refine our statistical analysis for our taxonomy.
More specifically, we need to compare each bug type one by one in addition to the comparison we have already done.

We consider this contribution to be 60\% complete.

\section{Publication Plan\label{sec:publication-plan}}

This section presents our planned publications over the course of the next years.

\begin{itemize}
	\item {\bf Publication 6}. {\tt PRECINCT} will be submitted to {\tt International Working Conference on Source Code Analysis and Manipulation, SCAM 2016}.
	\item {\bf Publication 7}. {\tt BIANCA} will be submitted to {\tt Journal of Software: Evolution and Process. 2016}.
	\item {\bf Publication 8}. {\tt RESEMBLE} will be submitted to {\tt International Conference Software Maintenance and Evolution, ICSME 2017}.
	\item {\bf Publication 9}. Our proposed bug taxonomy will be submitted to {\tt Empirical Software Engineering, ESE 2017}.
	\item {\bf Publication 10}. Our framework as a whole, {\tt BUMPER}, {\tt JCHARMING}, {\tt RESEMBLE} and {\tt BIANCA} will be submitted to {\tt Transaction of Software Engineering, TSE 2017}.
	\item {\bf Thesis}. In parallel to publications 9 and 10, I plan to write my Ph.D thesis.
\end{itemize}

Figure \ref{fig:planning} presents and overview of the publications planning and the relationship between the publications.

 \begin{figure}[h!]

 \centering
 \begin{ganttchart}
	  [
 inline
]{1}{30}
 \gantttitle{2014}{6}
 \gantttitle{2015}{6}
 \gantttitle{2016}{6}
 \gantttitle{2017}{6}
 \gantttitle{2018}{6}  \\
 \gantttitle{W}{2}
 \gantttitle{S}{2}
 \gantttitle{F}{2}
 \gantttitle{W}{2}
 \gantttitle{S}{2}
 \gantttitle{F}{2}
 \gantttitle{W}{2}
 \gantttitle{S}{2}
 \gantttitle{F}{2}
 \gantttitle{W}{2}
 \gantttitle{S}{2}
 \gantttitle{F}{2}
 \gantttitle{W}{2}
 \gantttitle{S}{2}
 \gantttitle{F}{2} \\


\ganttbar{$Courses$}{1}{4}

\\
\ganttbar{$JChar$}{5}{8}
\ganttbar{$JChar_2$}{12}{14}
\ganttbar{$Taxonomy$}{19}{22}
\\
\ganttbar{$Bumper$}{4}{11}
\ganttbar{$Pasmat$}{22}{25}
\\
\ganttbar{$Precinct$}{13}{16}
\\
\ganttbar{$Resemb$}{18}{20}
\ganttbar{$Thesis$}{25}{29}
\\
\ganttbar{$Bianca$}{15}{20}
\\


\ganttlink{elem0}{elem1}
\ganttlink{elem1}{elem2}
\ganttlink{elem2}{elem3}
\ganttlink{elem4}{elem2}
\ganttlink{elem4}{elem3}
\ganttlink{elem4}{elem6}
\ganttlink{elem6}{elem7}
\ganttlink{elem6}{elem9}

\ganttlink{elem6}{elem5}
\ganttlink{elem7}{elem5}
\ganttlink{elem9}{elem5}
\ganttlink{elem3}{elem5}
\ganttlink{elem5}{elem8}

 \end{ganttchart}


 \caption{Provisional Publication Planning\label{fig:planning}}
 \end{figure}

%!TEX root = research_proposal.tex

\chapter{Conclusion\label{chap:conclusion}}

The maintenance and evolution of complex software systems account for more than 70\% software's life cycle.
Hundreds of papers have been published with the aim to improve our knowledge of these processes in terms of issue triaging, issue prediction, duplicate issue detection, issue reproduction and co-changes prediction.
All these publications gave meaning to the millions of issues that can be found in open source issue \& project and revision management systems.
Context-aware IDE and think tank in open source architecture (\cite{chansler2011architecture}) open the path to approaches that support developers during their programming sessions by leveraging past indexed knowledge and past architectures.

In this research proposal, we first presented the most influential papers in the different fields our work lies on in Chapter \ref{chap:relwork}.
Chapter \ref{chap:methodology} presented our proposal in details while chapter \ref{chap:plan} detailed our attempt planning.

More specifically, in Chapter \ref{chap:methodology}, we presented four approaches: {\tt BUMPER}, {\tt JCHARMING}, {\tt RESEMBLE}, {\tt BIANCA}. Also, we proposed a taxonomy of bugs. When combined into {\tt pErICOPE} (Ecosystem Improve source COde during Programming session with real-time mining of common knowlEdge), these tools (i) provide the possibility to search related software artifacts using natural language, (ii) accurately reproduce field-crash in lab environment, (iii) recommend improvement or completion of current block of code and (iv) prevent the introduction of clones / issues at commit time.

{\tt BUMPER} has been designed to handle heavy traffic while {\tt JCHARMING} can reproduce 85\% of real-world issues we submitted to it. On its side {\tt BIANCA} is able to flag 41.5\% and 48.89\% of commit introducing a bug as dangerous with 13.4\% and 21\% of false positive using two code different code normalization, respectively.

Our future works, according to our publication plan described in section \ref{sec:publication-plan}, are as follows. First, we want to improve the performances of {\tt BIANCA} in terms of false positives.
Then, create the IDE plugin that will support {\tt RESEMBLE}. Finally, we want to refine our taxonomy by including as many as datasets as possible.



\addcontentsline{toc}{chapter}{Bibliography}
\bibliography{library, russel}  %place your .bib files here
\bibliographystyle{alpha}                   %the bibliography style to use

\end{document}
